

\lstset{language=Java, caption=Repairing strategy example, label = snippet:exCode1}
\begin{figure}[t]
\begin{lstlisting}
void foo(){
	String s = Input();
	String st = s.substring(1, 61);	
	if(st.length() == 1){..}
	if(st.length() < 39){..}
	if(st.startsWith(">:")){..}
	if(st.contains(Input())){..}
	st1 = st1.substring(1, 2);
}
\end{lstlisting}
\end{figure}

\lstset{language=Java, caption=Repairing strategy Instrumentation example, label = snippet:exCode2}
\begin{figure}[t]
\begin{lstlisting}
void foo(){
	String s = Input();
	try{
		String st = s.substring(1, 61);
	}
	catch(IndexOutOBoundException ex){
		st = ">:ffpsfkdvine%2&*&jd";
		 EncounterRepair.encounterRepairSet("<foo()>", str2);
	}
	ConstraintStorageMapDynamic.updateMinLength("<foo()>", st, 1);
	ConstraintStorageMapDynamic.updateMaxLength("<foo()>", st, 1);
	st = GenerateStringDynamic.init("<foo()>", st);	
	if(st.length() == 1){..}
	ConstraintStorageMapDynamic.updateMaxLength("<foo()>", st, 39);
	st = GenerateStringDynamic.init("foo()>", st);
	if(st.length() < 39){..}
	ConstraintStorageMapDynamic.updatePrefixSet("<foo()>", st, ">:");
	st = GenerateStringDynamic.init("<foo()>", st);
	if(st.startsWith(">:")){..}
	String temp = Input();
	ConstraintStorageMapDynamic.updateContainsSet("<foo()>", st, temp);
	st = GenerateStringDynamic.init("<foo()>", st);
	if(st.contains(temp)){..}
	Character c = null;
	try{
	 c = st1.chatAt(1);
	}
	catch(IndexOutOfBoundException ex){
	c = st1.failSafeCharAt(1,st1.length());
	}
}
\end{lstlisting}
\end{figure}

The code in~\ref{snippet:exCode1} is an example to demonstrate how the repairing strategy
works. Code snippet given in~\ref{snippet:exCode2} is the instrumented patched version of
the earlier. From the example~\ref{snippet:exCode1}, we can see that there are multiple 
conditional statements which represents the constraint on the string object \code{st} which
involves three constraints which can be evaluated statically and one is dependent on some
input string. The input function is represented by~\code{Input()} method. Statically we
collected all the bugs and evaluated the string as \code{st = ">:ffpsfkdvine\%2\&*\&jd";} 
For the constraints like \code{st.contains(Input())}, which can not be evaluated statically
can only be collected and resolved at runtime. In dynamic constraint collection and evaluation 
phase, we did instrumentation which also collected all the static constraints those evaluated
earlier to make sure at the time of string evaluation all the constraints get evaluated.
\code{ConstraintStorageMapDynamic.updatePrefixSet("<foo()>", st, ">:");}, this particular
method invocation is responsible to accumulate the constraint information associated
to a string object. The instrumented statement \code{st = GenerateStringDynamic.init("<foo()>", st);}
is responsible for generating new string object which complies to the constraints 
associated to it.




