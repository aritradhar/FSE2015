
\section{Results}
\label{sec:results}


\subsection{Experimental Setup}
\label{sub:experimentalSetup}

We have looked to several bug repositories like Buzilla, Apache issue tracker,
Eclipse project issue tracker etc and noticed number of bugs with major,
minor, citical and blocker priorities. We took choose several bugs from them
considering couple of facts
\begin{mylist}

\item \textbf{Popularity}: How much it is used among the developers and
industries,
\item \textbf{Severity} : We consider major, crtical and blocker priority bugs
only considering the facts at the bug affects dependent applications and other
libraries severely.
\item \textbf{Age and state} : We consider latest bugs which were reported in
the last 5 years.
We also consider such bugs which still remains un-fixed.

We did all the experiments in a laptop pc equiped with a dual core intel i5
2.3-2.9 GHz processor, 8 GB or RAM, Microsoft Windows 8.1 operating system, JDK
version 1.7-45 with 2 GB of allocated heap space. All the bug reproduction was done on
Eclipse Juno IDE. For static analysis and instrumentation we have used \soot\
version 2.5.0 and \soot\ \infoflow\ for static taint analysis. We have also used
java decompiler JD version 0.7.0.1. 

\end{mylist}


\subsection{Evaluation Matrices}
\label{sub:evaluationMartices}

We have conducted the evaluation based on the matrices which emparically
measures the precision and the performance of the developed tool. We measures
the precesion in terms of the quality of the patch and some other criteria
listed bellow and the performance based on the time taken and the memory
footprint.

\begin{mylist}

\item \textbf{Patch similarity with the developers' patch} : After the
patching we compared both the autogenerated patch code with the developer's patching code
we found in te bug repositories. In the case the bug is still un-fixed we looked
for the comments and discussion in the pannel and collects the information
about the potential patch. At the first step, we visually compared the patches
to see how much they are similar and dissimilar with the developers' patch. We
than use the instrumented class files and place it to the library archieve
replacing the buggy class files. We reproduced the similar test cases of the
bugs and used the automated patched version and lated fixed version of the same
libray to compare the results. In case the results are the primitive or string
type we have printed the output in the console and compare them. In case the
output is some complex object we compare the properties of them. Apart from the
test cases to reproduce the bug, we also ran couple of good test case to make
sure that the patch is not behaving any other way. Based on this expriemnt we
made a matrics named \textbf{Patch Similarity Index (PSI)} which can contains
three values, high, medium and low where we consider high PSI being a good close
quality patch.

\item \textbf{Autogenerated patch size and the develops' patch size} :

\item \textbf{Already handled exception} :

\item \textbf{Cascaded exception} :

\item \textbf{Time} :

\item \textbf{Memory Consumption} :
\end{mylist}


\begin{table*}[t]
\centering
\scriptsize
\begin{tabular}{l|c|c|r|r|r|r|r|r}
\multicolumn{1}{c|}{\textbf{Library/Application}} &
\multicolumn{1}{c|}{\textbf{Priority}} &
\multicolumn{1}{c|}{\textbf{Repairing Result}} &
\multicolumn{1}{c|}{\textbf{$LOC$}} & 
\multicolumn{1}{c|}{\textbf{$IC_O$}} &
\multicolumn{1}{c|}{\textbf{$IC_{UO}$}} &
\multicolumn{1}{c|}{\textbf{Time}} &
\multicolumn{1}{c|}{\textbf{Memory}} &
\multicolumn{1}{c}{\textbf{Cascading}} \\

\hline
\code{Apache Commons}   	  & Major 	& Success &  & & & & & \\
\code{Apache Aries} 	 	  & Major 	& Success &  & & & & & \\
\code{Apache HttpClient} 	  & Major 	& Success &  & & & & & \\
\code{Apache Log4j} 		  & Major 	& Success &  & & & & & \\
\code{Apache Hive} 			  & Major 	&  		  &  & & & & & \\
\code{Apache Struts2} 		  & Major 	& Success &  & & & & & \\
\code{Eclipse AspectJ} 		  & Major 	& Success &  & & & & & \\
\code{Apache Commons Lang} 	  & Major 	& Success &  & & & & & \\
\code{Apache Commons Math} 	  & Major 	& Success &  & & & & & \\
\code{Apache Commons Net} 	  & Major   & Success &  & & & & & \\
\code{Apache servicemix-soap} & Major   & Success &  & & & & & \\
\code{Apache Qpid} 			  & Blocker & Success &  & & & & & \\
\code{Apache Pivot} 		  & Major   & Success &  & & & & & \\
\code{Apache XalanJ2} 		  & Major 	& Success &  & & & & & \\
\code{Apache SOAP} 			  & Major 	& Success &  & & & & & \\
\code{Apache Commons VFS} 	  & Major 	& Success &  & & & & & \\
\code{Apache Commons Compress}& Blocker & Success &  & & & & & \\
\code{Apache Commons CLI1.x}  & Major 	& Success &  & & & & & \\
\code{Apache Commons CLI2.x}  & Major 	& Success &  & & & & & \\
\code{Apache Wicket} 		  & Major 	& Success &  & & & & & \\
\code{Apache Wicket} 		  & Major 	& Success &  & & & & & \\


\end{tabular}
\caption{Experimental results}
\label{tab:results}
\end{table*}
