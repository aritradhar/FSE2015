\section{Conclusion and Future Work}
\label{sec:conc}

Running programs may crash unexpectedly due to vulnerabilities in the code and
malformed data.
The cost associated with such crashes can be vary with the criticality of the
applications. Therefore, it is hardly a surprise that automatic program
repairing has been an actively researched area over the past decade.
In this work, we have presented a novel program repairing technique, and a tool
\tool\ based on it which employs hybrid program analysis to protect a running
program from failures originating from string-handling errors leading to a
program crash.
Our choice of \java\ \code{String} APIs is driven mainly by the popular usage of
string objects and bugs associated with them.
By focusing on a specific data type, and taking the program context into
account, \tool\ can develop patches that are precise and  semantically close to
the ones developed by the developers.
Hence, when the patches are activated, the program exhibits a behavior which is
close to the intended program behavior.

%this part is now in discussion section
Our study shows that \tool\ can handle programs that are real, and can produce
patches efficiently. Motivated by the results of our study as well as by the
research conducted by other researchers, we intend to extend \tool\ in the
future by adding support for other \java\ APIs and also by adding more
intelligence to the process of patch generation.

\paragraph{\textsc{Acknowledgements.}} %\todo{Fill it.}
We would like to thank \soot\ mailing list community, specially Steven Arzt
who helped with our queries related to various issues while implementing \tool.
