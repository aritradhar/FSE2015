

\section{Related Works}
\label{sec:relatedWorks}

\subsection{Recent Works on Data Structure Repairing}
\label{subsec:RecWorksDataStructure}

Automated data-structure repairing techniques are there in the litarature for a
while. In the papers~\cite{DBLP:conf/oopsla/DemskyR03},
\cite{Demsky03automaticdata},~\cite{DBLP:conf/icse/DemskyR05},
\cite{DBLP:conf/issre/DemskyR03},~\cite{DBLP:conf/issta/DemskyEGMPR06} the
authors mostly concentrated on specific data-structures like \emph{FAT-32},
\emph{ext2}, \emph{CTAS} (a set of air-traffic control tools developed at the
NASA Ames research center) and repairing them. The authors represented a
specification language by which they able to see consistency property these
data-structure. Given the specification, they able to detect the inconsistency
of these data-structures and repair them.
The repairing strategy involves detecting the consistency constraints for the
particular data structure, for the violation, they replace the error condition
with correct proposition. In the paper~\cite{DBLP:conf/icse/DemskyR05}, the
authors proposed repair strategy by goal-directed reasoning. This involves
translating the data-structure to a abstract model by a set of model definition
rules. The actual repair involves model reconstruction and statically mapped it
to a data structure update. In their paper~\cite{DBLP:conf/oopsla/2007} authors
Elkarablieh et al. proposed the idea to statically analyze the data structure to
access the information like recurrent fields and local fields. They used their
technique to some well known data structures like singly linked list, sorted
list, doubly liked list, N-ary tree, AVL tree, binary search tree, disjoint set,
red-black tree, Fibonacci heap etc.

\subsection{Works on Software Patching}
\label{subsec:RecWorksSoftPatch}

In their paper~\cite{DBLP:conf/sosp/PerkinsKLABCPSSSWZER09}, authors Jeff H.
Perkins et al. presented their \emph{Clear view} system which works on windows
x86 binaries without requiring any source code. They used invariants analysis for
which they used Daikon~\cite{DBLP:journals/scp/ErnstPGMPTX07}. They mostly
patched security vulnerabilities by some candidate repair patches.

Fan Lon et al in their paper~\cite{DBLP:conf/pldi/LongSR14} presented their new
system \emph{RCV} which recovers applications from divide-by-zero and
null-deference error. Their tool replaces \emph{SIGFPE} and \emph{SIGSEGV}
signal handler with its own handler. The approach simply works by assigning
zero at the time of divide-by-zero error, read zero and ignores write at the time
of null-deference error. Their implementation was on $x86$ and $x86-64$
binaries and they also implemented a dynamic taint analysis to see the effect of their
patching until the program stabilizes which they called as \emph{error
shepherding}.

\subsection{Genetic Programming, Evolutionary Computation}
\label{subsec:RecWorksGeneric}

Reserch works on program repair based on genetic programming and evolutionary
computation can be found in the paper of Stephanie Forrest et
al.~\cite{DBLP:conf/gecco/2009g} and Westley Weimer et
al~\cite{DBLP:journals/cacm/WeimerFGN10} respectively. In the papers, the
authors used genetic programming to generate and evaluate test cases. They used
their technique on the well known Microsoft Zune media player bug causing tme
to freeze up.


