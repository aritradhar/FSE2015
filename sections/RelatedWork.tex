\section{Related Work}
\label{sec:relatedWork}

Automated program repairing has been an active
area over past decade. The approaches that have been proposed by the researchers
broadly fall into two categories namely, static and dynamic.

\subsection{Static Approaches}
\myparagraph{Counter-example Driven.}
Some of the static approaches work based on a counter-example or a violated
invariant that is reported from the field. These approaches then repair the
program by automatically developing a patch and then ensuring its correctness
using computationally intensive techniques such as model-checking
\cite{biere2014, wei-issta-2010}. Static techniques are effective in producing accurate patches.
However, shutting down the system to produce and apply patches is not always a feasible or
a desirable option. The motivation for our technique comes from the fact that for some
applications fixing the bug after a program crash is not an option.

\myparagraph{String Generation.}
A key to effective program repairing is to reduce the problem by 
targeting a specific domain, or specific data structures such as arrays, or data types such as strings, and then use
specialized transformation and solving techniques for repairing which can exploit the constrained environment.
Gulwani \cite{Gulwani:2011} considered Microsoft Excel programs as the use case scenario and
identified string transformation as the major class of programming problem.
The author designed an algorithm for learning a string expression
that is consistent with input-output examples. An input-output example is
generated from a mapping which maps a set of strings to a string defining a
operation such as concatenation.
Singh and Gulwani \cite{Singh:2012} deal with semantic transformation of a string that need 
to be interpreted as more than a sequence of
characters. Their approach may complement our approach by
improving our string generation process.

Samimi et al. present a tool \textit{PHPQuickFix} based on the static analysis approach and that
targets a special case of printing related bug in HTML \cite{SamirniSAMTH12}. Compared to their technique our
approach addresses much wider range of problems related to Java strings and can be extended to
other systems that have support for exceptions.
Tatlock et al. focus on the \java\ database API and related query string
\cite{Tatlock:2008}. They
propose a solution of type errors which is caused due to a type mismatch in
the database and the type assigned in the program. The authors also propose a
solution for refactoring where changing a class name associated with some
queries will reflect all the strings system wide. This approach is based on type-checking
and addresses different class of repairing problems.

\myparagraph{Static Data Repair.}
Demsky et al. propose repair strategy by goal-directed reasoning~\cite{conf/icse/DemskyR05}. This involves
translating the data-structure to an abstract model by a set of model definition
rules. The actual repair involves model reconstruction and statically mapping it
to a data structure update. Elkarablieh et al. propose an idea to statically analyze a data structure to
access the information like recurrent fields and local fields ~\cite{conf/oopsla/2007}. They use their
technique to some well known data structures like singly linked list, sorted
list, doubly linked list, N-ary tree, AVL tree, binary search tree, disjoint set,
red-black tree, and Fibonacci heap. Overall, these approaches are complementary to
our approach. 

\subsection{Dynamic Approaches}
Static approaches cannot exactly predict when the failure can happen. However,
dynamic approaches have a complete visibility to the program under execution and can
have a precise knowledge of its state. In order to overcome some limitations of static approaches
and exploit the execution knowledge, several promising dynamic approaches have
been proposed.

\myparagraph{Dynamic Data Repair.}
Demsky and Rinard present techniques that mostly concentrate on specific
data-structures like \emph{FAT-32}, \emph{ext2}, \emph{CTAS} (a set of
air-traffic control tools developed at the NASA Ames research center) and
repairing them~\cite{Demsky03automaticdata, conf/issre/DemskyR03,conf/oopsla/DemskyR03,
conf/issta/DemskyEGMPR06}. The authors present a specification language using which they
detect and check consistency properties in these data-structures and repair them.
For a violation, they replace the error condition with correct proposition.
These approaches develop either suboptimal patches or
isolate the data structure that is damaged which allows at least part of the
system to be functional. In contrast,
our approach tries to provide support to the entire application in the context of string-related failures
by trying to keep the system behavior
as close as possible to the intended. The advantage of these approaches is that they are
light-weight and like our approach, can fix a system on-the-fly
potentially allowing some sub-optimal behavior for a finite time until the
systems self-stabilizes.

Long et al. ~\cite{conf/pldi/LongSR14}
present a technique that repairs a crashing system \emph{RCV} on-the-fly.
The technique deals with two types of system violations, namely, divide-by-zero and
null-deference errors. Their tool replaces \emph{SIGFPE} and \emph{SIGSEGV}
signal handler with its own handler. The approach works by assigning zero
at the time of divide-by-zero error, read zero and ignores write at the time of
null-deference error. Their implementation is on $x86$ and $x86-64$ binaries.
They also implement a dynamic taint analysis to see the effect of their
patching until the program stabilizes which they called as \emph{error
shepherding}. However, unlike our approach, this technique requires a special runtime support which is 
built into the operating system.

\paragraph{Genetic Programming.}
The researchers have also tried genetic programming approach for repairing.
Goues et al. use genetic programming technique which is a stochastic search
method inspired by biological evolution \cite{GouesNFW12}. The technique generates program patch
by using already existing test cases to deal with bugs such as infinite loop, null
string, segmentation fault, and buffer overflow. Similar approach has been presented by 
Weimer et al. \cite{WeimerFGN10}. Goues et al. present study to understand the real life feasibility and
the fraction of the bugs their genetic programming-based tool can repair as well as the cost associated
with it \cite{GouesDFW12}. These approaches are test-driven and rely on the availability and completeness of test-suites.
In contrast, our approach is completely independent of a test-suite.

\paragraph{Strings Related.}
Samimi et al. present a dynamic repairing technique for PHP applications  \cite{SamirniSAMTH12}.
They present a tool \textit{PHPRepair} to repair auto-generated
malformed HTML codes from the PHP scrips. Often the HTML codes do not have
proper tags which are silently corrected by the browser but the result is
different across browsers. The authors employ an efficient SAT solver
using cost optimization to find an efficient repair.

\paragraph{Other Dyanamic Approaches.}
Perkins et al. present a system named \emph{Clear view} which works on windows x86 binaries
without requiring any source code ~\cite{conf/sosp/PerkinsKLABCPSSSWZER09}.
They use invariants analysis for which they
use Daikon~\cite{DBLP:journals/scp/ErnstPGMPTX07}. They mostly patched security
vulnerabilities by some candidate repair patches. Unlike our approach, their approach requires a test-suite
to develop the knowledge about the invariants.

\subsection{Main Contrasting Features of Our Approach.}
Our approach targets only string objects for repairing allowing it
generate highly precise and intelligent program patches which generate very few or none
cascading exceptional events and produces a program behavior which is very close
to the expected behavior under the event of crashing.
In addition, our approach is hybrid with a heavy static component which enables
all the analysis including the side-effect analysis based on a taint analysis to
perform dynamically. It incurs negligible overhead even in the event of
crashing. Moreover, \tool\ works at the application level, and hence is easily portable.
Many of these proposed repairing approaches have been found to be effective on
various problems that are scoped appropriately. We believe that ours is the first
generic approach that targets string objects and repairs program on-the-fly by
generating precise patches using the properties of string objects and leveraging contextual
information.