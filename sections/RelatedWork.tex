\section{Related Work}
\label{sec:relatedWork}

There has been considerable amount of research done in the area of automated
program repairing. The approaches that have been proposed by the researchers
broadly fall into two categories namely, static and dynamic.

The static approaches work based on the counter-example or the violated
invariants that are reported from the field. These approaches then repair the
program by automatically developing a patch and then ensure its correctness
using computationally intensive techniques such as model-checking
\cite{biere2014, wei-issta-2010}.
These techniques are effective in producing accurate patches. However, shutting
down the system to produce and apply patches is not always feasible or
desirable. To overcome these problems, several promising dynamic approaches have
been proposed. These approaches typically develop either suboptimal patches or
isolate the data structure that is damaged which allows at least part of the
system to be functional \cite{conf/issre/DemskyR03, conf/icse/DemskyR05,
conf/issta/DemskyEGMPR06}. The advantage of these approaches is that they are
light-weight and can fix the system on-the-fly. Long et al.
\cite{conf/pldi/LongSR14} have developed an approach that deals with two most
commonly observed software errors, and then suppressing the errors with the help
of a runtime that operates by first invoking a  signal handlers, and then by
running a dynamic symbolic execution to ensure no side-effects.
This approach is light-weight and like our approach fixes the errors on-the-fly
potentially allowing some sub-optimal behaviour for a finite time until the
systems self-stabilizes.
In contrast our approach targets only string objects for repairing allowing it
generate highly precise program patches which generate very few or none
cascading exceptional events and produces a program behavior which is very close
to the expected behavior under the event of crashing.
In addition, our approach is hybrid with a heavy static component which enables
all the analysis including the side-effect analysis based on a taint analysis to
perform dynamically. It incurs negligible overhead even in the event of
crashing.

%%added new
In the litarature there exists proir art where the authors used string
transformation and solving technique for repairing purpose. In the paper
\cite{Singh:2012}, the authors deals with semantic transformation of the string
like manipulating strings that need to be interpreted as more than a sequence of
characters, e.g., as a column entry from some relational table, or as some
standard data-type like date, time, currency, etc. In \cite{Gulwani:2011}, the
author designed a learning algorithm for learning a string expression that is
consistent with input output examples. The input output example is generated
from a mapping which maps a set of string to a string defining a operation like
concatenation. There are works on genetic programming technique like
\cite{LeGoues:2012Ex, LeGoues:2012, DBLP:journals/cacm/WeimerFGN10} where the
 technique generated program patch by using already existing test cases to deal
with bugs like infinite loop, null string, segmentation fault, buffer overflow
etc.

\ignore {
% % added from the mail : from PLDI author response

Automated repair of HTML generation errors in PHP applications using string
constraint solving

In this paper the authors proposed a technique to repair the auto generated
malformed HTML codes from the PHP scrips. Often the HTML codes do not have
proper tags which are silently corrected by the browser but the result is
different across browsers. The authors employed an efficient SAT solver named
Kodkod using cost optimization to find the best repair.
%%%%%%%%%%%%%%%%%%%%%%
Deep Typechecking and Refactoring

Here the authors focused on the java database API and related query string. They
proposed the solution of type errors which is caused due to the type mismatch in
the database and the type assigned in the program. The authors also proposed a
solution for refactoring where changing a class name associated with some
queries will reflect all the strings system wide.
%%%%%%%%%%%%%%%%%%%%%%%%
GenProg: A Generic Method for Automatic Software Repair

The author used genetic programming technique which is a stochastic search
method inspired by biological evolution. The technique generated program patch
by using already existing test cases to deal with bugs like infinite loop, null
string, segmentation fault, buffer overflow etc.
%%%%%%%%%%%%%%%%%%%%%%%%%
Automatic Program Repair with Evolutionary Computation

Same paper as the above (Journal version) written by same authors.
%%%%%%%%%%%%%%%%%%%%%%%%%%%%%
A Systematic Study of Automated Program Repair: Fixing 55 out of 105 Bugs for $8
Each

Extended work of the above. The paper deals with the real life feasibility if
GenProg like what fraction of the bugs it can repair and the cost associated
with it.
%%%%%%%%%%%%%%%%%%%%%%%%%%%
Automating String Processing in Spreadsheets Using Input-Output Examples

The author considered Microsoft Excel programs as the use case scenario and
identified string processing as the major class of programming problem which
includes names/phone-numbers/dates from one format to another, data cleansing,
extracting data from several text files or web pages into a single document,
etc. The author designed a learning algorithm for learning a string expression
that is consistent with input output examples. The input output example is
generated from a mapping which maps a set of string to a string defining a
operation like concatenation.
%%%%%%%%%%%%%%%%%%%%%%%%%%%%
Learning Semantic String Transformations from Examples

The authors deals with semantic transformation of the string like manipulating
strings that need to be interpreted as more than a sequence of characters, e.g.,
as a column entry from some relational table, or as some standard data-type like
date, time, currency, etc.


% %%%%%%%%%%%%%%%%%%%%%%%%%%%%%%%%%%%%%%%%%%%%%%%%%%%%%%%%%%%%%%%%%%%%%%%%%%%
Several approaches have been proposed in the past to ensure that programs can
recover from failures. Some of the approaches are based on static repairing
where the patches are synthesized automatically based on the counter examples
found in the field \cite{wei-issta-2010}.
However, it is not always desirable to shut down the system for the post-mortem
analysis and then relaunch it after fixing the defect. In order to overcome this
weakness, dynamic approaches have been proposed to deal with problems that are
related to memory, data, and incorrect programming constructs such as infinite
loops \cite{Carbin:2011, KlingMCR12, conf/sosp/PerkinsKLABCPSSSWZER09}. Some of
the approaches work either by identifying and isolating damaged data or memory
portions \cite{conf/issre/DemskyR03, conf/icse/DemskyR05,
conf/issta/DemskyEGMPR06}, or by delaying the execution until the program
self-stabilizes \cite{Eom:2012}, or by finding the alternative execution paths
\cite{PezzeRWZ11}, or by disabling suppressing signals and hoping that the
program can recover automatically from the errors \cite{conf/pldi/LongSR14}.
Static approaches strive for correctness whereas dynamic approaches are
typically optimistic and work on the assumption that some suboptimal behavior
under certain conditions is acceptable.

\myparagraph{Data Structure Repairing}
% \label{subsec:RecWorksDataStructure}
Demsky and Rinard have proposed approaches that repair data structures ~\cite{
Demsky03automaticdata, conf/issre/DemskyR03,conf/oopsla/DemskyR03,
conf/issta/DemskyEGMPR06} the authors mostly concentrated on specific
data-structures like \emph{FAT-32}, \emph{ext2}, \emph{CTAS} (a set of
air-traffic control tools developed at the NASA Ames research center) and
repairing them. The authors represented a specification language by which they
able to see consistency property these data-structure.
Given the specification, they able to detect the inconsistency of these
data-structures and repair them.
The repairing strategy involves detecting the consistency constraints for the
particular data structure, for the violation, they replace the error condition
with correct proposition. In the paper~\cite{conf/icse/DemskyR05}, the authors
Demsky et al. proposed repair strategy by goal-directed reasoning. This involves
translating the data-structure to a abstract model by a set of model definition
rules. The actual repair involves model reconstruction and statically mapped it
to a data structure update. In the paper~\cite{conf/oopsla/2007} authors
Elkarablieh et al. proposed the idea to statically analyze the data structure to
access the information like recurrent fields and local fields. They used their
technique to some well known data structures like singly linked list, sorted
list, doubly liked list, N-ary tree, AVL tree, binary search tree, disjoint set,
red-black tree, Fibonacci heap etc.

\myparagraph{Works on Software Patching}
% \label{subsec:RecWorksSoftPatch}
In their paper~\cite{conf/sosp/PerkinsKLABCPSSSWZER09}, authors Perkins et al.
presented their \emph{Clear view} system which works on windows x86 binaries
without requiring any source code. They used invariants analysis for which they
used Daikon~\cite{DBLP:journals/scp/ErnstPGMPTX07}. They mostly patched security
vulnerabilities by some candidate repair patches.

Fan Long et al. in their paper~\cite{conf/pldi/LongSR14} presented their new
system \emph{RCV} which recovers applications from divide-by-zero and
null-deference error. Their tool replaces \emph{SIGFPE} and \emph{SIGSEGV}
signal handler with its own handler. The approach simply works by assigning zero
at the time of divide-by-zero error, read zero and ignores write at the time of
null-deference error. Their implementation was on $x86$ and $x86-64$ binaries
and they also implemented a dynamic taint analysis to see the effect of their
patching until the program stabilizes which they called as \emph{error
shepherding}.

\myparagraph{Genetic Programming, Evolutionary Computation}
% \label{subsec:RecWorksGeneric}
Reserch works on program repair based on genetic programming and evolutionary
computation can be found in the paper of Forrest et al.s~\cite{conf/gecco/2009g}
and Weimer et al.~\cite{DBLP:journals/cacm/WeimerFGN10} respectively. In the
papers, the authors used genetic programming to generate and evaluate test
cases. They used their technique on the well known Microsoft Zune media player
bug causing the application to freeze up.
}


