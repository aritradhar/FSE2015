\section{Overview}
\label{sec:overview}

Ensuring correctness of a program statically is an undecidable problem. Thus
there is always a tradeoff between precision and scalability that static program
analysis must balance. Static analysis achieves high scalability by making sound
approximations, which typically leads to false positives. Complex
programming logic and data coming from diverse sources make the already hard
problem worse. As a result, successful execution of a real application can never
be guaranteed, and unexpected failures may happen. These failures often result
in applications throwing runtime exceptions, which if not handled correctly
may crash the application.

\ignore{ \begin{table}[t] \scriptsize \centering
\begin{tabular}{|l|r|r|}
\hline
\multicolumn{1}{|c|}{\textbf{Runtime Exception Type}} &
\multicolumn{1}{c|}{\textbf{Frequency}} & \multicolumn{1}{c|}{\textbf{\%}}\\
% \scalebox{0.83}
% {
\hline
\code{NullPointerException} & $34912$ & $54.94$ \\
\code{ClassCastException} & $7504$ & $11.81$ \\
\code{IndexOutOfBoundsException} & $6637$ & $10.44$ \\
\code{SecurityException}  & $5818$ & $9.15$ \\
\hline
\end{tabular}
\caption{Prominent runtime exceptions from
\texttt{stackoverflow}~\cite{stackoverflow}.}
\label{tab:stackoverlow}
% }
\end{table}
}


\lstset{language=Java, caption=Snippet from \code{fileUtils} class in Apache
Commons library. , label = snippet:exampleRepairing2}
\begin{figure}[t]
\centering
\begin{lstlisting}
public static String getPathNoEndSeparator
        (String filename) {
  return doGetPath(filename, 0);
}
private static String doGetPath
        (String filename, int separatorAdd) {
  if(filename == null) return null;
  int prefix = getPrefixLength(filename);
  if (prefix < 0) return null;
  int index = indexOfLastSeparator(filename);
  if ((prefix >= filename.length()) || (index < 0))
        return "";
  return filename.substring(prefix,
        index + separatorAdd);
}
\end{lstlisting}
\end{figure}

Code~\ref{snippet:exampleRepairing2} corresponds to methods from
\code{fileUtils} class in Apache Common IO library. The method
\code{getPathNoEndSeparator()} throws a \code{StringIndexOutOfBounds} exception,
which originates from statement \code{return filename.substring(prefix, index +
separatorAdd)} on line 13 when the method is called with parameter
\code{"/foo.xml"}.  Here, the value of \code{prefix} as returned by the method
\code{getPrefixLength} is 1. It fails to satisfy the constraint implied by the
program condition \code{prefix <= index + separatorAdd} for \code{substring}
method, which ensures that \code{beginIndex} cannot be greater than
\code{endIndex}. As a result, the exception is thrown.


\lstset{language=Java, caption=Patch for \code{fileUtils} bug in Apache Commons
library., label = snippet:exampleRepairing3, firstnumber = 13}
\begin{figure}[t]
\centering
\begin{lstlisting}
String temp = null;
try {
  temp = filename.substring(prefix, index + separatorAdd);
} catch(IndexOutOfBoundsException ex) {
  int length = filename.length;
  int t = index + separatorAdd;
  temp = filename.substring(
    getStart(prefix,t,length), getEnd(prefix,t,length));
}
return temp;
\end{lstlisting}
\end{figure}

A closer inspection of this code snippet shows that the string variable
\code{filename} invokes two methods, namely \code{length} and \code{substring}
on lines $11$ and $13$ respectively. \java\ \code{String} API documentation
specifies that \code{length} does not throw any runtime exceptions. The only
exception that this invoke statement can throw is when the receiver object
referenced by \code{filename} is \code{null}. However, the check on line $7$
indicates that this situation would not arise. The method \code{substring} may
throw \code{IndexOutOfBoundsException} exception that can potentially crash the
program. A good patch to handle this failure should take into account all of
these observations. 

Code~\ref{snippet:exampleRepairing3} presents the patch automatically generated
by \tool. This patch replaces the invoke statement on line 13 in
Code~\ref{snippet:exampleRepairing2}, which is now wrapped within a
\code{try--catch} block. The \code{catch} corresponding to
\code{IndexOutOfBoundsException}
% is added on line 15
ensures that control passes
to the catch block only when the exception is thrown. Line 20 shows two method
calls namely \code{getStart} and \code{getEnd} that are inserted by \tool. These
methods, using the knowledge about the length of \code{filename} acquired with
the help of the code on line 17, compute legally correct indexes required by
\code{substring} method to satisfy the constraint related to \code{beginIndex}
and \code{endIndex}. Method \code{substring} can now regenerate the substring
ensuring that the method call will not fail. The actual patch provided by the
developers is semantically similar to the one developed by \tool, and both
versions of the program generate the exact the same output.
% for the failed test case.

\ignore{Similarly, the patch developed by \tool\ for the bug depicted in
Code~\ref{snippet:exampleRepairing1} is semantically similar to the actual one
provided by the developers and is presented in
Code~\ref{snippet:exampleRepairing4}. Here the object referenced by the string
variable \code{temp} is regenerated after adjusting the offset and ensuring that
the constraint represented by the program condition \code{offset <= length}
would never be violated.


\lstset{language=java, caption=Patch for the Apache Log4j bug.,
label = snippet:exampleRepairing4, firstnumber =4}
\begin{figure}[t]
\centering
\begin{lstlisting}
try {
    temp = new String(chars, offset, length);
} catch(StringIndexOutOfBoundsException ex) {
    int i = chars.length;
    temp = new String(chars,
        IndexRepair.getStart(offset, length, i),
            IndexRepair.getEnd(offset, length, i));
}
priorVariables.add(temp);
\end{lstlisting}
\end{figure}
}

\ignore{\note{\tool\ performs hybrid constraint analysis to produce high-quality
patches.
This analysis ensures that the generated \code{String} objects obeying constraints imposed on them
by the program so as to exhibit a behavior that is close to the intended one. This ability of
\tool\ is illustrated by an example in
Code~\ref{snippet:constraintCollection}. Line numbers $3, 4, 5 $ and $7$ in
Code~\ref{snippet:constraintCollection} contain conditions associated with
\code{st}. The first three constraints can be collected and evaluated
statically and the last one dynamically as \code{userInput()} will return a
\code{String} object only at runtime. All of these constraints are stored in a
data structure called \textit{constraint store} for evaluation which is shown in
Figure~\ref{fig:constraint}. %\tool\ uses constraint store to evaluate
%which is described in Algorithm~\ref{algo:constraint}. More detailed process is
%discussed later in~\xref{sec:tool:stage2}.
The process of evaluation is described in Algorithm~\ref{algo:constraint} and
discussed in~\xref{sec:tool:stage2}. }
\note{Aritra: Can't we provide a real example here? It need not be an
 example from our study. Any real example will do.}
}

\section{Problem Definition}
\label{sec:definition}

Let the behavior $\mathcal{B}$ of program $\mathcal{P}$ for input $\mathcal{I}$
be a sequence of data values $\textless b_1, \ldots, b_n \textgreater$ shared
with the environment, where these values may be used to print information on
screen, access and manipulate files and databases, and exchange data with other
processes or threads. For brevity, we assume that the program is sequential.
However, our formalization and arguments can be extended to multi-threaded
programs and their behaviors.

Consider the behavior $\mathcal{B}$ to be composed of $\mathcal{B}_c = \textless
b_{1_c}, \ldots, b_{n_c} \textgreater$ and $\mathcal{B}_n = \textless b_{1_n},
\ldots, b_{n_n} \textgreater$, where elements in $\mathcal{B}_c$ consist of
critical values that form core functionality of the program, while elements in
$\mathcal{B}_n$ are noncritical values with respect to the core functionality of
the program.
% We consider behaviors of the two programs, $\mathcal{P}$ and
% $\mathcal{P}'$, equivalent if their critical components are identical for same
% program inputs. Formally, $\mathcal{P} \equiv \mathcal{P}' \iff$
% ($\mathcal{B}_c
% = \mathcal{B}'_c \land \mathcal{I} = \mathcal{I}'$). In other words, for
% equivalent program behaviors their noncritical behaviors do not matter.
%  
% Let $\mathcal{B}_{I_f}$ be the behavior for $\mathcal{P}$ under failure input
% $\mathcal{I}_f$. and let the data element that correspond to the failure does
% not belong to $\mathcal{B}_{{I_f}_c}$. Let the element be $b_{m_n}$.
% 
% Let p be an element of P which results in a failure under cer-
% tain input I f . Let the intended behavior of P under this input be
% BI f , and let the data element that correspond to the failure does not
% belong to BI f c . Let the element be bmn .
%  
% In this work, we restrict our approach to data values that are strings.
% 
If $\mathcal{B}_{\theta}$ is the behavior for $\mathcal{P}$ under failure input
$\theta$, then our approach attempts to
% i) find a minimal $p$ that was associated with the failure,
% ii) ensures that $b_{m_n} \not\in \mathcal{B}_{{I_f}_c}$,
develop a program patch to convert $\mathcal{P}$ to $\mathcal{P}'$,
% while ensuring that:
such that 
% \begin{mybullet}
%  \item
 $\mathcal{P}'$ does not violate the core behavior of $\mathcal{P}$ under
failure, \ie\ $\mathcal{B}'_{\theta_c}$ is same as the intended behavior
$\mathcal{B}_{\theta_c}$.
% , and
%  \item
Note that resulting behavior $\mathcal{B}'_{\theta_n}$ may not be
equivalent to $\mathcal{B}_{\theta_n}$.
% is close to the intended
% behavior $\mathcal{B}_{\theta_n}$. In other words, the difference between
%  $\mathcal{B}'_{\theta_n}$ and 
% $\mathcal{B}_{\theta_n}$ is within limits acceptable to the programmer.
% \end{mybullet}

\ignore{the reviewers are not happy with the informal notion of "close to" or "acceptable".
The changes I have made still do not address this issue convincingly. If we cannot
come up with a good solution to this problem, we might say
that formal definition of acceptance criteria and its enforcement is future work. Please
comment on this. 
}

\ignore{\note{I removed the offending text. Does it read better? -- md. Made
another minor change. - Rahul}}

% \subsection{Goals}
% \label{sec:tool:goals}

In this work, we restrict our approach to string data values. We identify the
broad design goals for a technique to automatically repair malformed strings or
incorrect handling of strings as follows:

% i) identifies the statements which might be vulnerable to string-related
% errors,
% and are less critical to the functionality of the application such that
% suboptimal behavior might be acceptable,
% iii) generates patches by identifying constraints on the string data and if
% required, tweaks \code{String} API  parameters to regenerate legally correct
% string data,
% iv) optimizes the number of statements to be patched by retaining only the
% ones
% that need to be protected,

\myparagraph{(i) High patch fidelity} We require that the patched program must
preserve the intended program behavior, \ie\ the patch must be precise, and
should not induce any undesirable control flows in the repaired program. 
This goal naturally follows from the problem definition. However, we set two
more goals associated with the security and performance of the technique.

\myparagraph{(ii) Non-invasive instrumentation} We require that the technique
must ensure no side-effects (aside from optimally repairing objects) during
normal program execution, and activate patches only when the program is
guaranteed to crash.

\myparagraph{(iii) Low system overhead} We desire that the patched program must
incur no runtime overhead during normal program execution, and only negligible
overhead in case of failures.

% We next present in detail techniques and algorithms to produce patches under
% more complex scenarios.
% Our study presented in \xref{sec:results} suggests that majority of
% the string generation scenarios in practice are less complex.
