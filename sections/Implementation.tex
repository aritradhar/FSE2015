\section{Implementation}
\label{sec:implementation}

We implemented a prototype of \tool\ as described in \xref{sec:design} for
repairing runtime exceptions originating from unhandled \java\ \code{String}
APIs. Our end-to-end toolchain is completely automated and was written in
$\sim$$12.7K$ lines of \java. We leveraged the \soot~\cite{soot} framework for
bytecode analysis and instrumentation, and \infoflow~\cite{infoflow} for static
taint analysis.
We now briefly describe a few salient features of our implementation, which is
also available for download at \url{http://goo.gl/d1zcXD}.

\subsection{Taint Analysis}

\infoflow\ performs its taint propagation over \code{Units}, which are \soot's
intermediate representation of the \java\ source code. We extended the
\infoflow\ framework to a) enable seamless coupling with \soot, and b) determine
whether it is safe to patch a given \soot\ \code{Unit}. Specifically, we added
a mapping that retrieves \code{Unit}s for statements to be patched given a
specified method signature. This is relevant since the same statement, say
\code{int x = 1;} has the exact same representation even if it appears more
than once in a same method. We also added a utility method to determine if a
\code{Unit} must be patched if it lies along the path between a source and sink 
in the call graph (as generated by \soot).

\subsection{Call graph analysis}
\label{subsec:callChainLookUp}

\tool\ leverages \soot\ generated call graph to determine both inter- and
intra-method checked runtime exceptions (recall \xref{sec:tool:stage1}). \soot\
uses the \code{Trap} class to manage exception handling for both classes of
exceptions discussed above. Each \code{Trap} object has start, end and handler
unit.  We tagged every \code{Unit} in a \code{HashMap} if it belonged to an
existing \code{Trap}, so as to exclude it from instrumentation during the
repairing phase.

\subsection{Constraint Analysis}
\label{subsec:constraint analysis}

\tool\ makes a forward pass over the \code{Units} identified by the taint
analysis and other program analyses in the first phase to gather constrains over
string literals of interest (recall \xref{sec:tool:stage2}), and builds a
\code{HashMap} of \code{ConstraintDataType}, a custom data type to store and
evaluate these constrains. Specifically, each \code{ConstraintDataType} entry
stores four key parameters---the permissible prefixes, substrings, minimum and
maximum length---that specify constraints corresponding to a \code{String}
literal.

Constraint evaluation over these \code{ConstraintDataType} entries is done as
discussed earlier in Algorithm~\ref{algo:constraint}. However, if the gathered
constraints can not be satisfied statically, \eg\
\code{if(str.contains(userInput()))}, \tool\ instruments the bytecode before the
conditional statement with a static invocation to i) populate the corresponding
\code{ConstraintDataType} entry, and ii) recompute the permissible values of the
string object with already existing constraints (see Code
snippet~\ref{snippet:exCode2}).


\subsection{Optimizations}
\label{subsec:optimizations}

\tool\ performs a few other optimizations to improve the precision and quality
of the patches.

\subsubsection{Minimize constraint analysis}
\label{subsubsec:minimizeConstrintInstrumentation}

\tool\ collects constraints only for those string literals that may be involved
in a runtime exception. For example, if a string object does not involve API
methods that can throw runtime exception, then it is not required to collect and
evaluate constraints on them. This significantly reduces the number of
statements analyzed for instrumentation.
 
\begin{table*}[t]
% \setlength{\tabcolsep}{3pt}
\centering
\caption{\tool's accuracy results when applied to $30$ bugs in popular
open-source libraries.}
\caption*{
\scriptsize
\centering
\setlength{\tabcolsep}{3pt}
\begin{tabular}{ll|ll|ll}

$\mathcal{N}_{CG}$ & \# nodes in call graph & $\mathcal{T}$ & \# total cases in
test suite & $FCI$ & Flow Consistency Index\\

$\mathcal{N}_{Unit}$& \# \code{Units} analyzed & $\mathcal{F}_{P}$ & \# failed
tests in patched version w/o forced patching & $\mathcal{IC}_{NO}$
& Instrumentation w/o optimization (recall \xref{subsec:optimizations})\\

$\mathcal{S}_{U}$ & \# successful tests in unpatched & $\mathcal{F}_{P}^{*}$ &
\# failed tests in patched version w/ forced patching & $\mathcal{IC}_{WO}$ &
Instrumentation w/ optimization (recall \xref{subsec:optimizations})\\

$\mathcal{F}_{U}$ & \# failed tests in unpatched & $PPI$ & Patch Precision
Index & $\mathcal{RS}_{CE}$ & Cascaded exception exists
\end{tabular}
}

\scriptsize
\begin{tabular}{|l|c|l|r|r||r|c|r|c|c||r|c|r|r|c|}

\hline
\multicolumn{1}{|c|}{\textbf{API}} &
\multicolumn{1}{c|}{\textbf{BugID}} &
\multicolumn{1}{c|}{\textbf{Priority}} &
\multicolumn{1}{c|}{\textbf{$\mathcal{N}_{CG}$}} &
\multicolumn{1}{c||}{\textbf{$\mathcal{N}_{Unit}$}} &
%\multicolumn{1}{c|}{\textbf{$PPI$}} &
\multicolumn{1}{c|}{\textbf{$\mathcal{S}_{U}$}} &
\multicolumn{1}{c|}{\textbf{$\mathcal{F}_{U}$}} &
\multicolumn{1}{c|}{\textbf{$\mathcal{T}$}} &
\multicolumn{1}{c|}{\textbf{$\mathcal{F}_{P}$}} &
\multicolumn{1}{c||}{\textbf{$\mathcal{F}_{P}^{*}$}} &
\multicolumn{1}{c|}{\textbf{$PPI$}} &
\multicolumn{1}{c|}{\textbf{$FCI$}} &
\multicolumn{1}{c|}{\textbf{$\mathcal{IC}_{NO}$}} & %instrumentation with
%optimization
\multicolumn{1}{c|}{\textbf{$\mathcal{IC}_{WO}$}} & %instrumentation without
% optimization
\multicolumn{1}{c|}{\textbf{$\mathcal{RS}_{CE}$}}  % cascaded exception
%new added col
\\

\hline
\code{Aries} & \cite{ARIES1204} & Major &$3.5K$  & $129$ &
$18$ & $2$ & $20$ &  &  & $0.83$ &  & $42$ & $5$ &  \\

\code{Commons CLI1.x} & \cite{CLI193} & Critical & $3.2K$ & $53$ &
$14$ & $2$ & $16$ &  &  & $0.74$ & & $19$& $19$ &  \\

\code{Commons CLI2.x} & \cite{CLI46} & Major & $3.2K$ & $21$ &
$13$ & $3$ & $16$ & $3$ & $1$ & $0.62$ &$1$ & $13$ &$2$ & $\checkmark$\\

\code{Commons Compress} & \cite{COMPRESS26} & Blocker & $4.0K$ & $134$ &
$32$ & $1$ & $33$ &  &  & $0.74$ & & $46$& $4$ & \\

\code{Commons IO} & \cite{IO179} & Major & $3.3K$ & $125$ &
$27$ & $1$ & $28$ &  &  & $0.77$ & & $76$ & $1$ & \\

\code{Commons Lang} & \cite{LANG457} & Major & $5.1K$ & $240$ &
$16$ & $2$ & $18$ &  &  & $0.59$ & & $168$& $8$ &  \\

\code{Commons Math} & \cite{MATH198} & Major & $3.4K$ & $300$ &
$19$ & $2$ & $21$ & $2$ &  & $0.89$ &$1$ & $36$ &$2$ &  \\

\code{Commons Net} & \cite{NET442} & Major & $3.3K$ & $14$ &
$22$ & $1$ & $23$ &  &  & $0.84$ & & $6$ & $1$ & \\

\code{Commons VFS} & \cite{VFS338} & Major &$4.5K$ & $37$ &
$18$ & $1$ & $19$ &  &  & $0.65$ & & $20$ & $2$ &  \\

\code{Derby} & \cite{DERBY4748} & Major & $4.4K$ & $40$ &
$30$ & $2$ & $32$ &  &  & $0.46$ & & $47$ & $6$ &  \\

\code{Eclipse AJ Weaver} & \cite{EclipseBug432874} & Major & $20.6K$ & $50$ &
$17$ & $2$ & $19$ & $2$ & $2$ & $0.98$ & & $4$ & $1$ & $\checkmark$ \\

\code{Eclipse AJ} & \cite{EclipseBug333066} & Major & $25.0K$ &$39$ &
$14$ & $2$ & $16$ &  &  & $0.87$ & & $6$ & $1$ & \\

\code{FlexDK 3.4} &\cite{SDK14417} & Minor & $6.3K$ & $600$ &
$13$ & $2$ & $15$ &  &  & $0.74$ & & $207$ & $25$&  \\

\code{Hama 0.2.0} &\cite{HAMA212}  & Critical & $3.7K$ & $35$ &
$13$ & $1$ & $14$ &  &  & $0.55$ & & $28$ & $5$ &  \\

\code{HBase 0.92.0} &\cite{HBASE4481}  & Critical & $4.8K$ & $61$ &
$24$ & $1$ & $25$ &  &  & $0.83$ & & $13$ & $2$ &  \\

\code{Hive} &\cite{HIVE6986} & Trivial &$4.4K$ & $23$ &
$18$ & $1$ & $19$ &  &  & $0.75$ & & $8$ & $1$ &  \\

\code{HttpClient} &\cite{HTTPCLIENT150} & Major & $3.3K$ & $14$ &
$20$ & $3$ & $23$ &  &  & $0.89$ & & $6$& $1$ &  \\

\code{jUDDI} & \cite{JUDDI292} & Major &$3.2K$ & $70$ &
$28$ & $1$ & $29$ &  &  & $0.85$ & & $10$ & $2$ &  \\

\code{Log4j} & \cite{ApacheLog4jBug} & Major & $3.2K$ & $17$ &
$8$ & $3$ & $11$ &  &  & $0.74$ &  & $6$ &$1$ &  \\

\code{MyFaces Core} & \cite{MYFACES416} & Major  & $4.5K$ & $50$ &
$11$ & $3$ & $14$  &  &  & $0.83$ &  & $4$& $2$ &  \\

\code{Nutch} & \cite{NUTCH1547} & Major & $4.5K$ & $90$ &
$8$ & $3$ & $11$ &  &  & $0.68$ & & $8$ & $1$ &  \\

\code{Ofbiz} & \cite{OFBIZ4237} & Minor & $4.4K$ & $28$ &
$20$ & $3$ & $23$ & $3$ &  & $0.45$ &$1$ & $6$ &$1$ &  \\

\code{PDFBox} & \cite{PDFBOX467} & Major &$4.4K$ & $23$ &
$15$ & $3$ & $18$ &  &  & $0.87$ & & $8$ & $1$ &  \\

\code{Sling Eclipse IDE} & \cite{SLING3095} & Major & $4.5K$ & $58$ &
$5$ & $1$ & $6$ &  &  & $0.59$ &  &$39$ & $6$ &  \\

\code{SOAP} & \cite{SOAP130} & Major &$5.0K$ & $165$ &
$18$ & $3$ & $21$ & $3$ &  & $0.84$ & $1$ & $32$ & $5$ &  \\

\code{SOLR 1.2} & \cite{SOLR331} & Major & $11.0K$ & $200$ &
$12$ & $2$ & $14$ &  &  & $0.89$ &  & $25$ & $4$ &  \\

\code{Struts2} & \cite{WW650} & Major & $16.0K$ & $80$ &
$11$ & $2$ & $13$ &  &  & $0.76$ & & $25$ & $2$ &  \\

\code{Tapestry 5} & \cite{TAP51770} & Major & $6.2K$  & $71$ &
$17$ & $3$ & $20$ &  &  & $0.70$ & & $31$ &$5$ &  \\

\code{Wicket} & \cite{WICKET4387} & Major & $70.0K$ & $68$ &
$20$ & $3$ & $23$ &  &  & $0.81$ & & $16$ & $1$ & \\

\code{XalanJ2} & \cite{XALANJ836} & Major & $3.3K$ & $33$ &
$11$ & $3$ & $14$ &  &  & $0.72$ & &$13$ & $2$ & \\

\hline

\end{tabular}

\label{tab:results}
\end{table*}

\subsubsection{Minimize patch instrumentation}
\label{subsubsec:minimizePatchInstrumentation}

\tool\ makes a forward pass over all bytecodes to determine if a specific string
object is modified after it has been patched. If the object is not modified then
no further patching of statements capable of throwing \code{NullPointerException} exceptions is
required, since the constraint would have been satisfied in the beginning and it would be valid
as long as the variable is not changed. Similarly, when the API usage is same and none of 
the method parameters are changed, no further patching would be required. This reduces the 
total number of possible
instrumentations required. 
