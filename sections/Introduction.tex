\section{Introduction}
\label{sec:intro}

% marker
%\textcolor{red}{\textbf{Changes done}}\newline


\ignore{Modern software applications are large and complex. In addition, they
run in
diverse environments and get inputs from variety of data sources. As a result,
predicting safe behavior for them at runtime can be difficult.
Moreover, giving assurances about the quality of service becomes practically
impossible when the applications are developed using third party
libraries and components. Such applications often exhibit vulnerabilities that
can be exploited by providing malicious inputs. In addition, diverse data
sources
and complex constraints on them make it challenging for programmers to ensure
that all data elements are correctly validated and processed. Any deficiencies
in this process makes applications prone to failures. Such defects can be hard
to detect at the compilation-time, and irrespective of the validation techniques
used, some of them may go undetected and exist in the applications even when the
applications are in production.}

Software systems form a functional backbone of business organizations and their
successful operation determines
the success of the organizations. Therefore, there is hardly a surprise that
companies
invest substantial amount of time and effort in testing and verifying the
software products and applications
that they develop. However, in spite of this effort and the tools used to
ensure their safety and security, the software invariably carries subtle bugs,
the presence of
which is often evident only when the software responds to them by throwing an
exception and then crashing as a result. The cost of a crash varies considerably
depending on the degree
of the criticality of the software. This cost would depend on whether the crash 
occurred during production or testing.

\ignore{The cost of failures can vary considerably depending on the
mission-critical
nature of applications. In particular, it would be extremely undesirable for an
unmanned aerial vehicle on its mission to allow the control system to crash in
case of a failure. Instead a suboptimal functioning for a short while might be
acceptable until the system fully stabilizes.}

A crash that occurs during production would
have an adverse effect on the business of an organization running the software,
and also on the business
of the company that developed the software. The organisations normally have no
other option, but to wait for the
developers to fix the bug that resulted on the failure, and then relaunch the
software after it is repaired. The bug-fixing
itself may take variable amount of time and the time could be long and
unpredictable if the bug is related to a third-party
component of the software.


On the other hand,
crashes occurred during internal acceptance testing may hamper the 
testing progress, since the testers would waste their time logging few bugs at a
time and then waiting for the newer version with the fixes.
This would negatively impact the testing efficiency lowering the overall quality
of the software.
Similarly, crashes during beta testing of a software product if performed by
public might frustrate them 
resulting in the rejection of the product from the market.

In both of these scenarios, it would be helpful if a temporary program patch
that saves the program from crashing
and moreover, also guarantees \textit{acceptable} and close to the intended
behavior
can be applied to the software on-the-fly. Program failures that result in
crashes often originate from subtle
program bugs that are related to unusual program inputs, unexpected environment
changes, or specific
thread schedules since correctness for most of the normal and expected 


\ignore{The expectations about the quality of service would largely
depend on how a failure may impact the business. For example, a commercial
online store may not afford a crash while it is listing its products to a
customer. Such unpleasant experiences might result in customers moving to other
online stores making an adverse impact on the business. Similarly, a software
company launching a new product would expect it to be stable while the product
is undergoing beta-testing. Any crashes occurring at that time would result in
negative feedback from the users and loss in the business. To make the matter
worse, these
failures may occur in software components that do not possess critical
functionality, and hence, may even get less attention to their quality at the
time of development. Nevertheless, irrespective of the criticality of these
components, if the crash occurs it is equally undesirable.}


\lstset{language=Java , caption=Apache Log4j bug example.,
label=snippet:exampleRepairing1}
\begin{figure}[t]
\begin{lstlisting}
private int substitute() {
  if (priorVariables == null) {
    priorVariables = new ArrayList<String>();
    priorVariables.add(new String(chars, offset, length));
  }
}
\end{lstlisting}
\end{figure}

The crashes are always undesirable, but they are particularly annoying when they
arise from \textit{non-critical} modules
that are not related to the core functionality of the software. For example,
consider Code ~\ref{snippet:exampleRepairing1} which depicts a
bug that
existed in Apache Log4j library version 2.0-beta9~\cite{ApacheLog4jBug} and
crashed
the logging framework. It was reported as a major bug in
spite of the fact that it occurred in logging component. The object
\code{priorVariables} is a \code{List} of String. On line 4, there is no check
on the variables to ensure
that invariants such as \code{offset + length <= chars.length}, \code{offset >
0}, and \code{length
> 0} hold.

In case of such a failure, rather than allowing the application to crash,
organizations would
prefer to collect diagnostic information to identify the defect and proceed with
the execution at times
allowing the system to run sub-optimally for a while with the hope that it will
stabilize eventually and complete its operation, or 
reveal a few more bugs in the same run. During the production, as long as such
suboptimal
behavior is within acceptable limits, the program survival would get higher
preference. The bug can then be
fixed in the later version as soon as possible. This would be particularly
useful if shutting down the system is
either not a practical option or is a less desirable option than continuing with
a suboptimal behavior.

\ignore{Several approaches have been proposed in the past to ensure that
programs can
recover from failures. Some of the approaches are based on static repairing
where the patches are synthesized automatically based on the counter examples
found in the field \cite{wei-issta-2010}.
However, it is not always desirable to shut down the system for the post-mortem
analysis and then relaunch it
after fixing the defect. In order to overcome this weakness,
dynamic approaches have been proposed to deal with problems that are
related to memory, data, and incorrect programming constructs such as infinite
loops \cite{Carbin:2011, KlingMCR12, conf/sosp/PerkinsKLABCPSSSWZER09}. Some of
the approaches work either by identifying and isolating
damaged data or memory portions \cite{conf/issre/DemskyR03, conf/icse/DemskyR05,
conf/issta/DemskyEGMPR06}, or by delaying the execution until the program
self-stabilizes \cite{Eom:2012}, or by finding the alternative execution paths
\cite{PezzeRWZ11}, or by disabling suppressing signals and hoping that the
program can recover automatically from the errors \cite{conf/pldi/LongSR14}.
Static approaches strive
for correctness whereas dynamic approaches are typically optimistic and work on
the
assumption that some suboptimal behavior under certain conditions is
acceptable.}

Several approaches have been proposed in the past to automatically fix the
incorrect program behavior by generating program patches.
Some of the approaches are based on static repairing
where the patches are synthesized automatically with the help of counter
examples
found in the field \cite{wei-issta-2010}. These approaches strive to achieve
correctness.
However, it is not always desirable to shut down the system for the post-mortem
analysis and synthesis, and then relaunch it
after fixing the defect. In order to overcome this weakness,
dynamic approaches have been proposed to deal with problems that are
related to memory, data, and incorrect programming constructs such as infinite
loops \cite{Carbin:2011, KlingMCR12, conf/sosp/PerkinsKLABCPSSSWZER09}.
Dynamic approaches are optimistic and often produce patches that are suboptimal.
The approaches work on the
assumption that some suboptimal behavior is acceptable if that does not
interfere with the core functionality. Some of them provide
partial functionality correctly if the corrupted data structures can be isolated
from the rest of the system.

In general, for dynamic approaches for program repairing, narrowing the problem
is the key in providing an effective solution to it. In this work, we propose a
novel
approach, which is hybrid in the nature and
deals with the failures originated from either malformed strings or incorrect
handling of strings. The approach targets string variables for patching for the
following reasons.

Firstly, \java\ applications are typically built using libraries, and
\code{String} APIs
are commonly used in third party libraries. Common and diverse usage of strings
in programs is a significant source of errors. In order to understand \java\
string objects'
involvement in the exceptions thrown in case of failures, we mined the
repository of posts
on \texttt{stackoverflow}~\cite{stackoverflow}. We realised that out
of the total $60K$ posts in which the exception was thrown, about $33K$ posts
were
related to the \java string objects. This data indicates that strings play a
vital
role in generating exceptions and potentially crashing programs. This result
coupled with the potentially
heavy cost of program crashes motivated us to develop a hybrid technique for
automatic repairing of \java\ programs for failures related to \code{String}
APIs.\comment{Aritra: could you please take a look at this?
Is this number really that large? Have you checked that the call stack traces
that are printed
have string objects in it?}


Secondly, by targeting a specific type of data
the approach can develop more precise patches that not only save the running
program from crashing, but 
also try to preserve the intended behavior. In other words, we exploit our
knowledge about the string behavior, its well-defined API, and also the program
context to automatically synthesize 
effective patches that precisely identify the program location as well as the
data that triggered the failure, and 
repair the program as close as possible to the failure trying to prevent the
failure propagation.

Our approach first identifies program statements statically
that might be vulnerable to string-related failures, and then develops patches
by trying to identify origins of errors and constraints on the strings. It
further uses dynamic
analysis to improve the precision of the patches generated by the static
analysis. 

%We applied \tool\ to $x$
%hugely popular open-source libraries involving over $y$ million lines of
%code to patch $z$ high priority bugs resulting from unhandled runtime
%exceptions from \java\ \code{String} APIs.

This work makes following contributions:
\begin{mylist}

\item We present the design and implementation of \tool\
(\xref{sec:motivation}, \xref{sec:design} and \xref{sec:implementation}) that
 generates effective program patches to handle string-related errors. These 
patches get activated only in case of program failures during runtime, and save
program from crashing ensuring its acceptable behavior.

%by identifying error sources and solving constraints on string data mainly
%statically and
%partly dynamically. Since the technique takes into account constraints on
%the data, the generated patches have a higher chance of producing intended
%behavior.

% \item We present \tool\ (\xref{sec:implementation}), a tool that implements
%our approach
% and effectively generates program patches to handle string-related errors
% \cite{}.

\item We use a finite state machine (FSM) as a formalism (\xref{sec:design}) to
describe the behavior of \java\ \code{String} API, and apply it to drive the
generation of exception-specific patches.

\item We applied \tool\ to several hugely popular open-source libraries to patch
$30$ bugs, several of them rated critical or major, resulting from unhandled
runtime exceptions from \java\ \code{String} APIs. The results of our study
(\xref{sec:results}) indicate that \tool\ can effectively produce patches that
save programs from crashing due to failures originating from known bugs. The
study also gives insights into the characteristics of the commonly occurring
string problems.

\item Manual inspection of the \tool-generated patches reveals that in most
cases they are semantically similar to the ones produced by the developers in
the later versions. %In fact, on couple of occasions the patches generated by
%\tool\ were found to be more accurate.
Thus, \tool\ can also guide developers in the process of building patches for
the future versions.
\end{mylist}

% The organization of the paper is as follows. Section~\ref{sec:motivation}
% describes the problem and provides an example to illustrate the proposed
% approach. Section~\ref{sec:architecture} provides the architecture of \tool\
%and
% describes the analyses implemented by the tool.
%Section~\ref{sec:implementation}
% provides implementation details and Section~\ref{sec:evaluation} presents the
% results of the study. Finally, Section~\ref{ref:related work} presents the
% related work and Section~\ref{sec:conclusion} concludes with the outline of
%the
% future work.

We have made our source code and data sets available to the open source
community at \url{http://goo.gl/d1zcXD}.









