\section{Discussion and Future Work}
\label{sec:discussion}

%% In this section we discuss the weaknesses of \tool\ and also propose remedies
%% to overcome them.

\begin{mylist}
 
 \item \myparagraph{Focus on String APIs} In its present form, \tool\ is
primarily targeted towards repairing \code{String} objects and handling API
exceptions. While this may seem to be a limitation, we believe that \tool{}'s
strength lies in the fact that it mines contextual data about runtime exceptions
related to \code{String} objects, which helps development of intelligent
patches. Further, \tool{}'s technique is generic and can be ported to other
classes of \java\ APIs. This requires extensive study of the characteristics
and constraints of other object types. We leave this extension for the future.
%\note{How can this be done. I think we need to give some intuition. -- md}

 \item \myparagraph{Patch correctness} \tool\ attempts to generate precise
patches considering the program context that avoids cascading exceptions to a
great extent and producing the intended behavior in cases of failure. However,
it sill cannot give guarantees about elimination of cascading exception,
particularly when there are heavy object dependencies in the program. In the
future, we plan to support \tool\ with program invariants that would ensure
acceptable behavior. The invariants can be specified by a programmer or can be
automatically generated with the help of training runs. 
%\note{This is not clear at all. -- md}

 \item \myparagraph{Handling of limited constraints} \tool{}'s constraint data
store is easy to build as it captures limited number of fairly simple
\code{String} characteristics, which are subsequently used to generate patches
for \code{String} objects. This approach may not be adequate particularly if the
program contains a large number of complex constraints. The quality of \tool{}'s
patches would generally depend on the nature of its constraint solver, which is
pluggable. \tool{}'s current solver is simple and can efficiently handle a
limited number of simple constraints. A more sophisticated off-the-shelf solver
may improve the repair quality. However, the current evaluation of the tool on
several library APIs described in~\xref{sec:results} indicates that the
constraints that exist in practice are normally less complex and are limited in
number.

\end{mylist}