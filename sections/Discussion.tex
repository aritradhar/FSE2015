\section{Discussion}
\label{sec:discussion}

\subsection{Limitations}
\label{sec:discussion:limitation}

\tool's major limitation arises from the fact that it
is heavily directed towards repairing handling \code{String} objects and API
exceptions. While this may seem to be a limitation, we believe that \tool's
strength lies in the fact that it mines contextual data about runtime exceptions
related to \code{String} objects that helps development of intelligent patches.
Moreover, \tool's technique is generic and can be ported to any other class of
\java\ APIs.

\tool generates precise patches considering the program context which avoids
cascading exceptions to a great extent producing the intended behavior in case
of failures. However, it still cannot give guarantees about elimination of cascading
exceptions, particularly when there are heavy object dependencies in the program.

\note{The constraint data store used by \tool\ is very simple as it captures
limited number of \code{String} characteristics which is just sufficient to
generate \code{String} objects. This may create complication if the program
contains very complex constraints. But  we tested \tool\ on several library APIs
described in Section~\ref{sec:results} and multiple synthetic test cases
composed by us and found no problem.}

\subsection{Remedy}
\label{sec:discussion:remedy}

\note{In the previous section we discuss about the major limitation of \tool\
i.e. it is rather focused on only \code{String} objects. But one key point here
is that the limitation exists because of the current implementation of \tool.
This technique can be applied to objects other than \code{String} but requires
rigorous study to understand the characteristics (behavior and types of
exceptions thrown) associated to the APIs related to that object. \tool
calculates patch by analyzing constrains. Proper study is required to understand
the constraints associated to the object to apply \tool.}

\note{The quality of \tool's patches also depends on the nature of the
constraint solver, which is pluggable. A more sophisticated solver may improve
the quality of program repair, and we leave comparison of different solvers for
future work. }

\subsection{Future Works}
\label{sec:discussion:futureWorks}

Our study shows that \tool\ can handle programs that are real, and can produce
patches efficiently. Motivated by the results of our study as well as by the
research conducted by other researchers, we intend to extend \tool\ in the
future by adding support for other \java\ APIs and also by adding more
intelligence to the process of patch generation.
