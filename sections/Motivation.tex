\section{Overview}
\label{sec:overview}

Ensuring correctness of a program statically is an undecidable problem.
However, there is always a tradeoff between the precision and the scalability
that a static program analysis has to live with. In order to achieve a high
scalability static analysis works by making sound approximations, which
typically lead to numerous false positives. Complex programming logic and data
coming from diverse sources make the already hard problem worse.
As a result, successful execution of a real application can never be guaranteed,
and unexpected failures may happen. These failures often result in runtime
exceptions being thrown by the applications that are generally not handled and
the program crashes.

% The cost of these runtime failures can vary depending on the criticality of
% the applications and can be very high for mission-critical applications.

\ignore{ \begin{table}[t] \scriptsize \centering
\begin{tabular}{|l|r|r|}
\hline
\multicolumn{1}{|c|}{\textbf{Runtime Exception Type}} &
\multicolumn{1}{c|}{\textbf{Frequency}} & \multicolumn{1}{c|}{\textbf{\%}}\\
% \scalebox{0.83}
% {
\hline
\code{NullPointerException} & $34912$ & $54.94$ \\
\code{ClassCastException} & $7504$ & $11.81$ \\
\code{IndexOutOfBoundsException} & $6637$ & $10.44$ \\
\code{SecurityException}  & $5818$ & $9.15$ \\
\hline
\end{tabular}
\caption{Prominent runtime exceptions from
\texttt{stackoverflow}~\cite{stackoverflow}.}
\label{tab:stackoverlow}
% }
\end{table}
}

\ignore{
\java\ applications are typically built using libraries, and \code{String} APIs
are commonly used in third party libraries. Common and diverse usage of strings
in programs is a significant source of errors. We mined the repository of posts
on \texttt{stackoverflow}~\cite{stackoverflow} to understand common types of
exceptions thrown in case of failures. Table~\ref{tab:stackoverlow} lists the
most prominent exceptions with the share of higher than $5\%$. The second column
indicates the types of exceptions, whereas the third column indicates their
overall percentage share. We observe that strings can play a role in generating
all except \code{SecurityException}. This result coupled with the potentially
heavy cost of program crashes motivated us to develop a hybrid technique for
automatic repairing of \java\ programs for failures related to \code{String}
APIs.}

\ignore{
\java\ applications are typically built using libraries, and \code{String} APIs
are commonly used in third party libraries. Common and diverse usage of strings
in programs can be a significant source of errors. We mined the repository of posts
on \texttt{stackoverflow}~\cite{stackoverflow} to understand \java\ string objects'
involvement in the exceptions thrown in case of failures. We realised that out
of the total $60K$ posts in which the exception was thrown, about $33K$ posts were
related to the \java\ string objects. \comment{Aritra: could you please take a look at this?
Is this number really that large? Have you checked that the call stack traces that are printed
have string objects in it?}
This data indicates that strings play an important
role in generating exceptions that can potentially crash programs. This result
coupled with the potentially
heavy cost of program crashes motivated us to develop a hybrid technique for
automatic repairing of \java\ programs for failures related to \code{String}
APIs.
}


\lstset{language=Java, caption=Snippet from \code{fileUtils} class of Apache
Commons library. , label = snippet:exampleRepairing2}
\begin{figure}[t]
\centering
\begin{lstlisting}
public static String getPathNoEndSeparator
        (String filename) {
  return doGetPath(filename, 0);
}
private static String doGetPath
        (String filename, int separatorAdd) {
  if(filename == null) return null;
  int prefix = getPrefixLength(filename);
  if (prefix < 0) return null;
  int index = indexOfLastSeparator(filename);
  if ((prefix >= filename.length()) || (index < 0))
        return "";
  return filename.substring(prefix,
        index + separatorAdd);
}
\end{lstlisting}
\end{figure}

Code~\ref{snippet:exampleRepairing2} corresponds to methods from
\code{fileUtils} class of Apache Common IO library. The method
\code{getPathNoEndSeparator()} throws a \code{StringIndexOutOfBounds} exception
on Windows OS, which originates from statement \code{return
filename.substring(prefix, index + separatorAdd)} on line 13 when the method is
called with parameter \code{"/foo.xml"}.  Here, the value of \code{prefix} as
returned by the method \code{getPrefixLength} is 1. It fails to satisfy the
constraint implied by the program condition \code{prefix <= index +
separatorAdd} for \code{substring} method which  ensures that \code{beginIndex}
cannot be greater than \code{endIndex}. As a result, the exception is thrown.


\lstset{language=Java, caption=Patch for \code{fileUtils} class
from Apache Commons library bug., label = snippet:exampleRepairing3, firstnumber
=13}
\begin{figure}[t]
\centering
\begin{lstlisting}
String temp = null;
try {
  temp = filename.substring(prefix, index + separatorAdd);
} catch(IndexOutOfBoundsException ex) {
  int length = filename.length;
  int t = index + separatorAdd;
  temp = filename.substring(
    getStart(prefix,t,length), getEnd(prefix,t,length));
}
return temp;
\end{lstlisting}
\end{figure}

A closer inspection of this code snippet shows that the string variable
\code{filename} invokes two methods, namely \code{length} and \code{substring}
on lines 11 and 13 respectively. \java\ \code{String} API documentation
specifies that \code{length} does not throw any runtime exceptions. The only
exception that this invoke statement can throw is when the receiver object
referenced by \code{filename} is \code{null}. However, the check on line 7
indicates that this situation would not arise. However, method \code{substring}
can throw \code{IndexOutOfBoundsException} exception that can potentially crash
the program. A good patch to handle this failure should take into account all of
these observations. 

Code~\ref{snippet:exampleRepairing3} presents the patch automatically generated
by \tool . This patch replaces the invoke statement on line 13 in
Code~\ref{snippet:exampleRepairing2}. The invoke statement is now wrapped inside
the \code{try} block and a \code{catch} corresponding to
\code{IndexOutOfBoundsException} is added on line 15. This ensures that
control passes to the catch block only when the exception is thrown. Line 20
shows two method calls namely \code{getStart} and \code{getEnd} that are
inserted by \tool. These methods, using the knowledge about the length of
\code{filename} acquired with the help of the code on line 17, compute legally
correct indexes required by \code{substring} method to satisfy the constraint
related to \code{beginIndex} and \code{endIndex}. Method \code{substring} now
can regenerate the substring ensuring that the method call would not fail.
The actual patch provided by the developers is semantically similar to the one
developed by \tool\ and both versions of the program generate exactly the same
output for the failed test case.

Similarly, the patch developed by \tool\ for the bug depicted in
Code~\ref{snippet:exampleRepairing1} is semantically similar to the actual one
provided by the developers and is presented in
Code~\ref{snippet:exampleRepairing4}. Here the object referenced by the string
variable \code{temp} is regenerated after adjusting the offset and ensuring that
the constraint represented by the program condition \code{offset <= length}
would never be violated.


\ignore{

By performing taint analysis, \tool\ identifies that statement 13 in
Code~\ref{snippet:exampleRepairing2}
can be vulnerable to a failure and it might throw
\code{IndexOutOfBoundsException}
or  \code{NullPointerException}
which are not handled by any of its callers in the call-chain.
\tool\ then tries to capture constraints on the involved variable
\code{filename}.
The only constraint on \code{filename} is \code{!(filename == null)} which is
completely
static indicating that providing a patch to deal with \code{NulPointerException}
is
not required. With a priori understanding of exception types and the invariants,
associated with them the tool then only creates a patch for
\code{IndexOutOfBoundsException}
tweaking the API parameters as guided by the invariants to regenerate strings.
As a result, \tool\ replaces statement 13 with the code depicted in
Code~\ref{snippet:exampleRepairing3}. The statement 17 shows
two methods namely \code{getStart} and \code{getEnd}
which compute legally correct bounds required by \code{substring} method to
satisfy the invariant
\code{prefix <= index + separatorAdd}. Using these bounds \code{substring}
regenerates the string referenced by variable \code{temp}. 

The actual patch provided by the developers is semantically similar to the one
developed by \tool\ and both versions
of the code generate exactly the same output. Similarly, the patch developed 
by \tool\ for the bug depicted in Code~\ref{snippet:exampleRepairing1}
is semantically similar to the actual one provided by the developers and is
presented in
Code~\ref{snippet:exampleRepairing4}. Here the object referenced by the string
variable \code{temp} is regenerated after adjusting the offset and ensuring that
the invariant \code{offset <= length}
is not violated.

}

\lstset{language=java, caption=Patch for the Apache Log4j bug.,
label = snippet:exampleRepairing4, firstnumber =4}
\begin{figure}[t]
\centering
\begin{lstlisting}
try {
    temp = new String(chars, offset, length);
} catch(StringIndexOutOfBoundsException ex) {
    int i = chars.length;
    temp = new String(chars,
        IndexRepair.getStart(offset, length, i),
            IndexRepair.getEnd(offset, length, i));
}
priorVariables.add(temp);
\end{lstlisting}
\end{figure}


\subsection{Problem Definition}
\label{subsec:definition}

Let the behavior $\mathcal{B}$  of program $\mathcal{P}$ for input $\mathcal{I}$ be defined as a sequence of data values $\textless b_1,
 \ldots, b_n \textgreater$ shared with the environment such as the values that are used to print some information on screen,
access and manipulate files and databases, and exchanged with other processes or threads. For brevity, we assume that the program is sequential.
However, our formalisation and arguments can be extended to multi-threaded programs and their behaviors.

Let this sequence be decomposed into two components $\mathcal{B}_c  = \textless b_{1_c},
 \ldots, b_{n_c} \textgreater$ and $\mathcal{B}_n = \textless b_{1_n}, \ldots, b_{n_n} \textgreater$, where all the elements 
 of $\mathcal{B}_c$ can be identified as critical values that form core functionality of the program, where as 
 the elements of $\mathcal{B}_n$ can be marked as noncritical values with respect to the core functionality of the program.
 
 Let $\mathcal{P}'$ be a variant of $\mathcal{P}$. We consider behaviors of the two progams equivalent if their critical components are identical for same program inputs.
Formally, $\mathcal{P} \equiv \mathcal{P}' \iff$ ($\mathcal{B}_c \equiv \mathcal{B}'_c \land \mathcal{I} = \mathcal{I}'$). In other words,
for equivalent program behaviors their noncritical behaviors do not matter.
 
 Let $p$ be an element of $\mathcal{P}$ which results in a failure under certain input $\mathcal{I}_f$. Let the intended behavior of $\mathcal{P}$ under this input be
 $\mathcal{B}_{I_f}$, and let the data element that correspond to the failure does not belong to $\mathcal{B}_{{I_f}_c}$. Let the element be $b_{m_n}$. 
 
 In this work, we restrict our approach to data values that are strings. The approach attempts to i) find a minimal $p$ that was associated with the failure, ii) ensures that 
 $b_{m_n} \not\in \mathcal{B}_{{I_f}_c}$, iii) develops a program patch $p'$
 that is applied on-the-fly to $\mathcal{P}$ in place of $p$, converting $\mathcal{P}$ to $\mathcal{P}'$, iv) does not violate the behavior $\mathcal{B}_{{I_f}_c}$,
 and v) ensures that the resulting behavior $\mathcal{B}'_{{I_f}_n}$ is
 close to and ideally same as the intended behavior $\mathcal{B}_{{I_f}_n}$.


We next present in detail the techniques and the algorithms used in our analysis
that can produce patches to regenerate string variables under more complex
scenarios. Our study presented in \xref{sec:results} suggests that majority of
the string generation scenarios in practice are less complex.



%--------------------

\ignore{
\subsection{Problem and Challenges}
\label{subsec:problem}

In this work, we present a technique which automatically repairs programs
that otherwise would crash due to failures related to malformed strings or
incorrect handling of
the string.  The technique i) identifies
the statements which might be vulnerable to string-related errors,
and are less critical to the functionality of the application such that
suboptimal behavior might be acceptable, ii) operates in non-invasive mode by
ensuring no
side-effects during normal program execution and activating patches only when
the program is guaranteed to crash, iii) generates patches, iv) optimizes the
number of
statements to be patched,  v) keeps program behavior as close as possible to the
intended behavior, and vi) incurs no runtime overhead during normal program
execution
and only negligible overhead in case of failures.
}


\ignore{
In this work, we present a technique and its implementation to automatically
repair programs
that otherwise would crash due to failures related to malformed strings or
incorrect handling of
the string.  The technique i) identifies
the statements which might be vulnerable to string-related errors,
and are less critical to the functionality of the application such that
suboptimal behavior might be acceptable, ii) operates in non-invasive mode by
ensuring no
side-effects during normal program execution and activating patches only when
the program is guaranteed to crash, iii) generates patches by identifying
constraints on the string
data and if required, tweaks \code{String} API  parameters to regenerate legally
correct
string data, iv) optimizes the number of statements to be patched by retaining
only the ones that need to
be protected,  v) keeps program behavior as close as possible to the
intended behavior by developing precise patches, and vi) incurs no runtime
overhead during normal program execution and only negligible overhead in case of
failures.
}


\ignore{
In order to achieve the goals enumerated in \xref{subsec:problem}, we
develop
several techniques which are outlined below.

\paragraph{Identifying Program Statements.} We perform static taint analysis
to identify sensitive data which are leaving the system via database, network
stream, file stream or console. Providing patches to the
statements that manipulate this data would be undesirable, since
activation of the patches in case failures may allow altered sensitive data to
eventually
reach users. Hence, we only mark those program statements which do 
not manipulate these data.

\ignore{We have used \soot\ \infoflow\ framework
for static taint analysis. We have defined potential taint source and sinks in
the
configuration file. The \infoflow\ frameworks returns all the program
statements which lies in any of the program path from source and sink. We tag
these
statements as unsafe to patch.}

\paragraph{Noninvasive Patching.} In case a runtime exception that is thrown
by a statement as a result of a failure is already caught and handled in a
program,
we skip that statement from patching to avoid interfering with the results. Such
statements are identified by analyzing call-graphs and ensuring that no caller
method
in the call-chain handles the exception or its superclass. By embedding the
patches inside
\texttt{catch} blocks, we ensure that they do not get activated during normal
program execution.

\paragraph{Patch Generation.} We first perform an
intra-procedural static analysis
to identify constraints on the string objects under consideration. By
identifying
the type of exceptions that can be thrown in case of a failure, we
develop patches that
regenerate string objects by tweaking \java\ \code{String} API used in
the statements to regenerate legal string objects and by trying to solve the
constraints. In the latter case,
we evaluate the constraints statically if complete information is available at
the compilation-time. Otherwise, the analysis automatically generates
code that performs dynamic analysis to solve the constraints, and then
inserts this code in the generated patches.

\paragraph{Optimizing Instrumentation.} We perform reaching definitions
analysis to skip marked statements
if the string variables that are contained in the statements are already
patched, and the variables
are not redefined along any path that originates from the patched statement.
This analysis reduces
instrumentation points in a program.

\paragraph{Patch Precision.} The precision of a program patch is improved,
firstly, by targeting only strings
for patching which allows us to develop more specialized patches, secondly, by
patching programs very
close to the points of potential failures which avoids unnecessary patching of
other
unaffected variables and their potential
side effects, thirdly, by analyzing the types of exceptions that can be thrown
which
provides valuable insights into
origins of failures, and finally, by considering all the constraints
on the strings. This would result
in a program behavior closed to the intended one in case of a failure.

\paragraph{Reduced Overhead.} The side-effect of non-invasive patches is that
they do not interfere during
normal execution which results in no runtime overhead. Even when they get
activated in case of failures,
they still cause negligible overhead since we perform no analysis during runtime
except if required resolve the dynamic constraints.
As our study reveals~\xref{sec:evaluation} the constraints are typically few and
simple, making
the dynamic analysis light-weight.
}



















\ignore{
\subsection{Historical Context}
\label{subsec:historicalContext}

In recent past, we have seen couple of disastrous failure of critical military
and civilian infrastructure system due to system failure/crash which is results
of some very common runtime exceptions.

\begin{mylist}
  
  \item In USS Yorktown, complete failure in propulsion and navigation system by
  a simple divide-by-zero exception in flight deck database.
  
  \item AT\&T telephone network failure causing by one faulty switch causing ATC
  commutation blackout.
  
  \item Air-Traffic Control System in LA Airport lost communication with all 400
  airplanes caused by a system crash triggered by integer (32bit) overflow.
  
  \item Mars rover curiosity B-side computer memory overflow causing OS suspend
  and multiple restart.
  
  \item Trans-Siberian Gas Pipeline Explosion in 1982 by deliberate bugs in
  software controlled valves.
  
  \item Near-blackout of the national grid in Austria caused by faulty function
  call.
  
  
\end{mylist}


\subsection{Data from Stack Overflow Posts}
\label{subsec:stackoverflow}

\begin{table}[t]
\small
\begin{tabular}{l|r|r}
\multicolumn{1}{c|}{\textbf{Runtime Exception Type}} &
\multicolumn{1}{c|}{\textbf{Frequency}} & \multicolumn{1}{c}{\textbf{\%}}\\
% \scalebox{0.83}
% {
\hline
\code{NullPointerException} & $34912$ & $54.94$ \\
\code{ClassCastException} & $7504$ & $11.81$ \\
\code{IndexOutOfBoundsException} & $6637$ & $10.44$ \\
\code{SecurityException}  & $5818$ & $9.15$ \\
\code{NoSuchElementException} & $2392$ & $3.76$ \\
\code{ArithmeticException} & $2338$ & $3.67$ \\
\code{ConcurrentModificationException} & $1889$ & $2.97$ \\
\code{DOMException} & $1024$ & $1.61$ \\
\code{ArrayStoreException} & $279$ & $0.43$ \\
\code{MissingResourceException} & $277$ & $0.43$ \\
% \code{BufferOverFlowException} & $161$ & $0.25$ \\
% \code{NegativeArraySizeException} & $122$ & $0.19$ \\
% \code{BufferUnderFlowException} & $66$ & $0.1$ \\
% \code{LSException} & $64$ &  $0.1$ \\
% \code{MalformedParameterizedTypeException} & $38$ & $0.05$ \\
% \code{CMMException}  & $8$ & $0.01$ \\
% \code{FileSystemNotFoundException} & $6$ & $0.009$ \\
% \code{NoSuchMechanismException} & $3$ & $0.0045$ \\
% \code{MirroredTypesException} & $1$ & $0.0015$
\end{tabular}
\caption{Top $10$ runtime exceptions from stackoverflow~\cite{stackoverflow}.}
\label{tab:stackoverlow}
% }
\end{table}

We have analyzed data from stack overflow and we looked for \java\ runtime
exception which are discussed most frequently. In the
table~\ref{tab:stackoverlow}, the data we find is tabulated along with their
occurrences and percentages.

From the table it is clear that null pointer exception in \java\ is not only the
most frequent but also the most dominant runtime exception having share of more
than $50$.
}

\ignore{
\lstset{language=Java, caption=Bug reproduction of code
snippet\ref{snippet:exampleRepairing}, label = snippet:exampleRepairing}
\begin{figure}[t]
\begin{lstlisting}
String path = "/foo.xml";
String s = ApacheBug.getPathNoEndSeparator(path);
\end{lstlisting}
\end{figure}
}


\ignore{
\lstset{language=Java, caption=Example of repairing technique on Appache Commons
Library in \code{fileUtils}, label = snippet:exampleRepairing2}
\begin{figure}[t]
\begin{lstlisting}
public static String getPathNoEndSeparator(String filename) {
return doGetPath(filename, 0);
}
private static String doGetPath(String filename, int separatorAdd) {
if (filename == null)
return null;
int prefix = getPrefixLength(filename);
if (prefix < 0)
return null;
int index = indexOfLastSeparator(filename);
if ((prefix >= filename.length()) || (index < 0))
return "";
String temp = null;
try
{
temp = filename.substring(prefix, index + separatorAdd);
}
catch(IndexOutOfBoundsException ex)
{
int length = filename.length;
int t = index + separatorAdd;
temp = filename.substring(getI(prefix,t,length), getJ(prefix,t,length));
}

return temp;
}
\end{lstlisting}
\end{figure}
}