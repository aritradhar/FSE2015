\documentclass{sig-alternate}

%% =============================================================
\def\tool{\textsc{Clotho}\xspace}
%\def\papertitle{\tool: Program Repairing using Exception Types, Constraint
%Automata and Typestate}
\def\papertitle{Clotho: Saving Programs from Malformed Strings and
Incorrect String-Handling}
\def\pdfauthors{}
\def\paperkeywords{}
%% =============================================================

\usepackage{color}

\newcommand{\subparagraph}{}

% You can tweak clickable link colors here:
\definecolor{linkcol}{rgb}{0,0,1}
\definecolor{citecol}{rgb}{0,0.5,0}
\definecolor{urlcol}{rgb}{0.3,0,0}

% Make pdflatex use letter size --md
\setlength{\pdfpagewidth}{8.5in}
\setlength{\pdfpageheight}{11in}

\let\proof\relax
\let\endproof\relax

\usepackage{cite}
\usepackage{amsthm}
% \usepackage{amsmath}
% \usepackage{amsfonts}
\usepackage{ragged2e}
\usepackage{txfonts}
\usepackage{fancyhdr}
\usepackage{amssymb}
\usepackage{fancyvrb}
\usepackage{graphicx}
\usepackage{times}
\usepackage{pifont}
\usepackage[hyphens]{url}
\usepackage{xspace}
\usepackage{sty/algorithm2e}
%%\usepackage[belowskip=-10pt,aboveskip=5pt,small,labelfont=bf]{caption}
\usepackage[aboveskip=5pt,small,labelfont=bf]{caption}
\DeclareCaptionType{copyrightbox}
\usepackage[bookmarks=true,%
bookmarksnumbered=true,%
colorlinks=true,%
linkcolor=linkcol,%
citecolor=citecol,%
urlcolor=urlcol,%
hypertexnames=true,%
pdfpagelabels,
% draft
]{hyperref}

\hypersetup{%
  colorlinks=false,% hyperlinks will be black
  pdfborderstyle={/S/U/W 1}% border style will be underline of width 1pt
}

\usepackage{sty/multirow}
\usepackage{sty/flushend}
\usepackage{epstopdf}
\usepackage[small,compact]{sty/titlesec}
\usepackage[font=scriptsize]{subfig}
\usepackage{sty/wrapfig}

\usepackage{epstopdf}
\usepackage{fancybox}
\usepackage{listings}
\usepackage{xcolor}
%\usepackage[]{datetime}
%\usepackage{lipsum}

\newcommand{\ignore}[1]{}
% \usepackage{rotating}

%==========================================
%Added definitions
% \usepackage{amsthm}
\theoremstyle{definition}
\newtheorem{definition}{Definition}[section]
%==========================================
%rename listing
\renewcommand\lstlistingname{Code}
%==========================================

%==========================================
%rotate column title
\usepackage{booktabs}
% \usepackage{xparse}
% \NewDocumentCommand{\rot}{O{45} O{1em} m}{\makebox[#2][l]{\rotatebox{#1}{#3}}}
%==========================================

%==========================================
%Automatic row number in table
% \usepackage{array,etoolbox}
% \preto\tabular{\setcounter{magicrownumbers}{0}}
% \newcounter{magicrownumbers}
% \newcommand\rownumber{\stepcounter{magicrownumbers}\arabic{magicrownumbers}}
%==========================================
\SetAlFnt{\small}
\SetAlCapFnt{\small}
\SetAlCapNameFnt{\small}
\SetVlineSkip{0pt}

\setlength\floatsep{5pt}
\setlength\textfloatsep{5pt}
\setlength\intextsep{5pt}

\hypersetup{
pdfauthor = {},
pdftitle = {\papertitle},
pdfkeywords = {\paperkeywords},
pdfcreator = {LaTeX with hyperref package},
pdfproducer = {pdflatex}}

\definecolor{dkgreen}{rgb}{0,0.6,0}
\definecolor{gray}{rgb}{0.5,0.5,0.5}
\definecolor{mauve}{rgb}{0.58,0,0.82}

\lstset{%frame=tb,
  captionpos=b,
  language=Java,
  aboveskip=10pt,
  belowskip=0pt,
  abovecaptionskip=5pt,
  belowcaptionskip=5pt,
  numberblanklines=false,
  showstringspaces=false,
  columns=flexible,
  basicstyle={\scriptsize\ttfamily},
  numbers= left,
  numbersep=5pt,
  numberstyle=\scriptsize,
  %frame=single,
  numberstyle=\tiny\color{gray},
  keywordstyle=\color{blue},
  commentstyle=\color{dkgreen},
  stringstyle=\color{mauve},
  breaklines=true,
  breakatwhitespace=true
  tabsize=2,
  xleftmargin=2em,
  morecomment=[l]{//},
  escapeinside={<@}{@>},
}

\makeatletter
\lst@Key{countblanklines}{true}[t]%
    {\lstKV@SetIf{#1}\lst@ifcountblanklines}

\lst@AddToHook{OnEmptyLine}{%
    \lst@ifnumberblanklines\else%
       \lst@ifcountblanklines\else%
         \advance\c@lstnumber-\@ne\relax%
       \fi%
    \fi}
\makeatother

% Use a smaller font size for URLs:
\makeatletter
\def\url@myurlstyle{%
   \@ifundefined{selectfont}{\def\UrlFont{\small}}{\def\UrlFont{\small}}}
   \makeatother
\urlstyle{myurl}

% This adds ':' to the characters after which not to break URLs, and
% defines a smaller typewriter font. --cpk
% changed small to sf -- md
\def\UrlNoBreaks{\do:\do\(\do\[\do\{\do\<}%
\def\UrlFont{\small\ttfamily}
\def\UrlOrds{\do\*\do\~}%

% For referencing sections, Vern-style
\newcommand\xref[1]{\S~\ref{#1}}

% Black filled circles with white number on it
% See Comprehensive LaTeX Symbol List --cpk
\def\blackI{\ding{182}}
\def\blackII{\ding{183}}
\def\blackIII{\ding{184}}
\def\blackIV{\ding{185}}
\def\blackV{\ding{186}}
\def\blackVI{\ding{187}}

\def\first{({\it i})\xspace }
\def\second{({\it ii})\xspace }
\def\third{({\it iii})\xspace }
\def\fourth{({\it iv})\xspace }
\def\fifth{({\it v})\xspace }

% For notes to authors:
\newcommand{\note}[1]{{\textcolor{red}{[\textit{#1}]}}}

% Fine-tuning for table spacing. --cpk
\def\TblSpT{\rule[-1ex]{0pt}{0pt}}
\def\TblSpB{\rule{0pt}{2.5ex}}

% Squeezing out some space. --cpk
% http://www-h.eng.cam.ac.uk/help/tpl/textprocessing/squeeze.html
%
%\renewcommand\subfigtopskip{0pt}
%\renewcommand\subfigbottomskip{5pt}
%\renewcommand\subfigcapskip{0pt}
%\renewcommand\floatpagefraction{.9}
%\renewcommand\topfraction{.9}
%\renewcommand\bottomfraction{.9}
%\renewcommand\textfraction{.1}
%\setlength{\parskip}{0em}
%\frenchspacing


%%%%%%%%%%%%%%%%%%%%%%%%%%%%%%%%%%%%%%%%%%%%%%%%%
% \hyphenpenalty 10000
% \exhyphenpenalty 10000
\sloppy
%%%%%%%%%%%%%%%%%%%%%%%%%%%%%%%%%%%%%%%%%%%%%%%%%

\newcommand{\comment}[1]{{\color{red}#1}}
\newcommand{\code}[1]{\texttt{\scriptsize{#1}}}
\newcommand{\mycomment}[1]{}
\newcommand{\todo}[1]{\textbf{TODO: #1}}
\newcommand{\mytab}{~~~~}
\newcommand{\sspace}{~}
\newcommand{\etal}{\textit{et al.}}
\newcommand{\eg}{{e.g.,}}
\newcommand{\ie}{{i.e.,}}
\newcommand{\lno}[1]{{\tiny{\textbf{(#1)~~}}}}

\newcommand{\java}{\textsc{Java}}
\newcommand{\soot}{\textsc{Soot}}
\newcommand{\infoflow}{\textsc{InfoFlow}}
\newcommand{\sdn}{{SDN}}
\newcommand{\verifier}{\textsc{Verifier}}
\newcommand{\validator}{\textsc{Flow\_Consistency\_ Validator}}
\newcommand{\outputs}{{\mathbb O}}
\newcommand{\stream}{{\mathbb S}}
\SetEndCharOfAlgoLine{}

%% for IEEEtran
\def\tablename{Table}

\renewcommand{\thetable}{\arabic{table}}

\newcommand*{\refname}{Bibliography}

\def\yes{\ding{51}}
\def\no{\ding{55}}

%------------------------------------------------------------------------------
%                                Space savers.
%------------------------------------------------------------------------------
% This mylist environment indents items, and saves less space than the above.
\newcounter{myctr}
\newenvironment{mylist}{\begin{list}{(\textbf{\arabic{myctr}})}
{\usecounter{myctr}
\setlength{\topsep}{1mm}\setlength{\itemsep}{0.5mm}
\setlength{\parsep}{0.5mm}
\setlength{\itemindent}{0mm}\setlength{\partopsep}{0mm}
\setlength{\labelwidth}{-2mm}
\setlength{\leftmargin}{0mm}}}{\end{list}}

% Space saving List environment for itemizing.
\newenvironment{mybullet}{\begin{list}{$\bullet$}
{\setlength{\topsep}{1mm}\setlength{\itemsep}{0.5mm}
\setlength{\parsep}{0.5mm}
\setlength{\itemindent}{0mm}\setlength{\partopsep}{0mm}
\setlength{\labelwidth}{-2mm}
\setlength{\leftmargin}{0mm}}}{\end{list}}

\newcommand{\myparagraph}[1]{\noindent{\scshape \bfseries #1.}}

%% added to remove indentation in susbsubsection for IEEEtran
% \makeatletter
% \def\subsubsection{\@startsection{subsubsection}% name
%       {3}% level
%       {\z@}% indent (formerly \parindent)
%       {0ex plus 0.1ex minus 0.1ex}% before skip
%       {0ex}% after skip
%       {\normalfont\normalsize\textbf}}% style
% \makeatother

%------------------------------------------------------------------------------
%                               Fancy header setup.
%------------------------------------------------------------------------------
%
\pagestyle{fancyplain}
\lhead{}
\lfoot{}
\chead{}
\rhead{}
\cfoot{\thepage}
%%\rfoot{{\scriptsize \today}}
\renewcommand{\headrulewidth}{0pt}

%\setlength{\intextsep}{10pt plus 2pt minus 2pt}

%% Reducing margins further -- md
%%\addtolength{\hoffset}{-0.25in}
%%\addtolength{\textwidth}{0.5in}
%%\addtolength{\voffset}{-0.25in}
%%\addtolength{\textheight}{0.4in}

% The space-saving sledgehammer. -- md
%\renewcommand{\baselinestretch}{0.95}


% \pagenumbering{arabic}

\begin{document}

%permission block
% \conferenceinfo{ESEC/FSE'15,}{August 30 -- September 4, 2015, Bergamo, Italy}
% \CopyrightYear{2015}
% \crdata{978-1-4503-3675-8/15/08}

\toappear{}

\title{\papertitle}

\numberofauthors{4}

\author{
    Aritra Dhar\xrci \\%%[4pt]
%     \email{aritra.dhar@xerox.com}
    \and
    Rahul Purandare\iiit  \\%%[4pt]
%     \email{purandare@iiitd.ac.in}
    \and
    Mohan Dhawan\ibm\iiit  \\%%[4pt]
%     \email{mohan.dhawan@in.ibm.com}
    \and
    Suresh Rangaswamy\iiit  \\%%[4pt]
%     \email{suresh1317@iiitd.ac.in}
%     \\ [4pt]
%     
    \sharedaffiliation
        \affaddr{{\xrci Xerox Research Centre India}} &
        \affaddr{{\iiit IIIT Delhi}} &
        \affaddr{{\ibm IBM Research}} \\ %%[4pt]
        \affaddr{{Bangalore, India}} &
        \affaddr{{New Delhi, India}} &
        \affaddr{\ \ \ {New Delhi, India}}
}

\maketitle

\begin{abstract}
{
%\small

Software is susceptible to malformed data originating from untrusted sources.
Occasionally the programming logic or constructs used are inappropriate to
handle the varied constraints imposed by legal and well-formed data.
Consequently, softwares may produce unexpected results or even crash.


In this paper, we present \tool, a novel hybrid approach that saves such
softwares from crashing when failures originate from malformed strings or
inappropriate handling of strings. \tool\ statically analyses a program to
identify statements that are vulnerable to failures related to associated string
data. \tool\ then generates patches that are likely to satisfy constraints on
the data, and in case of failures produces program behavior which would be close
to the expected. The precision of the patches is improved with the help of a
dynamic analysis.


We have implemented \tool\ for the \java\ \code{String} API, and our evaluation
based on several popular open-source libraries shows that \tool\ generates
patches that are semantically similar to the %actual
patches generated by the
programmers in the later versions. Additionally, these patches are activated
only when a failure is detected, and thus \tool\ incurs no runtime overhead
during normal execution, and negligible overhead in case of failures.

}
\end{abstract}


\category{D.2.5}{SOFTWARE ENGINEERING}
{Testing and Debugging }
[Error handling and recovery]



\keywords
{
Automatic Program Repair, 
Program Analysis,
Strings}

\section{Introduction}
\label{sec:intro}


Developers invest a significant amount of time and human involvement in testing
and verification to make their software production ready. However, in spite of
this effort and the tools used to ensure its safety and security, the software
%invariably carries subtle bugs, which are often evident only when the software
%throws an exception and/or crashes entirely. The cost of a severe exception or a
%crash varies considerably depending on the criticality of the software, and
invariably carries subtle bugs, which are evident only when the software
throws an exception and crashes. The cost of a 
crash varies depending on the criticality of the software, and
whether it occurred during production or testing.

A software bug in production systems may result in huge monetary losses to the
tune of hundreds of millions of dollars for organizations running third-party
software~\cite{hp, amazon, hershey, nike}. Further, these organizations must
wait for the vendor to release a patch for the offending software, which may
take days or even weeks. If a major software bug strikes during the internal 
acceptance testing, it may significantly hamper the testing progress itself,
%thereby affecting the entire software release cycle, and negatively impact the
%testing efficiency. Additionally, the software testers may have to wait for the
thereby affecting the entire software release cycle.
Additionally, the software testers may have to wait for the
newer patched version before they resume the testing process. Lastly, any such
crash during a software's beta testing phase might frustrate the public
resulting in rejection of the product itself. In all of the above scenarios, it
would be extremely useful if a temporary program patch that not only saves the
%program from crashing and moreover, but also guarantees \textit{acceptable} (and
%close to the intended) behavior can be applied to the software on-the-fly.
program from crashing, but also guarantees \textit{acceptable} and
close to the intended behavior can be applied to the software on-the-fly.


\lstset{escapeinside={/*@}{@*/}, language=Java , caption=Apache Log4j bug
example., label=snippet:exampleRepairing1}
\begin{figure}[t]
\begin{lstlisting}
private int substitute() {
  if (priorVariables == null) {
    priorVariables = new ArrayList<String>();
    priorVariables.add(/*@\\@*/ new String(chars, offset, length));
  }
}
\end{lstlisting}
\end{figure}

Software failures that result in crashes often originate from subtle program
bugs that are related to unusual program inputs, unexpected environment changes,
or specific thread schedules. While crashes are always undesirable, they are
particularly annoying when they arise from \textit{non-critical} modules that
are not related to the core software functionality. For example,
Code~\ref{snippet:exampleRepairing1} depicts a bug in Apache Log4j library
version 2.0-beta9~\cite{ApacheLog4jBug} that crashed the entire logging
framework. It was reported as a major bug in spite of the fact that it occurred
in logging component. The object \code{priorVariables} is a \code{List} of
String. On line 4, there is no check on the variables to ensure that invariants
such as \code{offset + length <= chars.length}, \code{offset > 0}, and
\code{length > 0} hold.
%
In case of such failures, rather than allowing the application to crash,
organizations would prefer to collect diagnostic information to identify the
defect, and proceed with a sub-optimal execution run hoping that it will
eventually stabilize, or reveal a few more bugs.


Prior work~\cite{wei-issta-2010, Carbin:2011, conf/sosp/PerkinsKLABCPSSSWZER09,
conf/pldi/LongSR14} proposes several mechanisms to automatically fix incorrect
program behavior by generating program patches. These approaches either need a
complete system shutdown to apply a patch, or isolate the faulty part of a data
structure on the fly thereby limiting the functionality, or keep suppressing the
exceptions with a hope that a suboptimal behavior would be acceptable until the
application stabilizes.

In this work, we propose a novel hybrid approach that deals with failures
originating due to malformed strings, or incorrect handling of strings in \java\
%softwares. We target string objects for patching, in particular, for the
applications. We target string objects for patching, in particular, for the
following two reasons.
First, \java\ applications are typically built using libraries, and
\code{String} APIs are commonly used in third party
libraries~\cite{Kawachiya:2008:ARM:1449764.1449795, gc, techreport}.  In
order to understand the usage and potential involvement of \java\ string
objects in the application failures, we mined
\texttt{stackoverflow}~\cite{stackoverflow} for related posts. We observed that
almost $33$K out of $60$K posts contained \java\ string related exceptions,
indicating heavy string usage in programs.
Second, we exploit extensive domain knowledge about strings to
automatically synthesize high-quality patches.

% Added -- md
\tool\ performs precise static analysis to identify program locations that are
vulnerable to string-related failures, and also the contexts under which they
trigger a failure. This enables repairing the program close to the point of
failure and generating precise patches that take into account the constraints on
the string objects.
%% preventing the failure propagation.
\tool\ further uses dynamic analysis to improve the precision of the patches
generated by the static analysis.

We applied \tool\ to patch $30$ bugs, several of them rated critical or major,
resulting from unhandled runtime exceptions from \java\ \code{String} APIs in
various hugely popular open-source libraries. Our evaluation shows that \tool\ 
develops precise patches that are semantically close to the ones developed by
the developers.

This work makes following contributions:
\begin{mylist}

\item We present the design and implementation of \tool\ (\xref{sec:overview},
\xref{sec:design} and \xref{sec:implementation}) that automatically generates effective
program patches to handle string-related errors.

\item We use a finite state machine (FSM) as a formalism (\xref{sec:design}) to
describe the behavior of \java\ \code{String} API, and apply it to drive the
generation of exception-specific patches.

\item  Our evaluation (\xref{sec:results}) indicates that \tool\ can effectively
produce patches that save programs from crashing due to failures originating
from known bugs. The results also gives insights into the characteristics of the
commonly occurring string problems.

\item Manual inspection of \tool-generated patches reveal that in most cases
they are semantically similar to the ones produced by the developers in the
later versions. 
Thus, \tool\ can also guide developers in the process of building patches.
\end{mylist}

Our source code and data sets are available to the open source community at
\url{https://github.com/aritradhar/CLOTHO}.










\section{Overview}
\label{sec:overview}

Ensuring correctness of a program statically is an undecidable problem. Thus
there is always a tradeoff between precision and scalability that static program
analysis must balance. Static analysis achieves high scalability by making sound
approximations, which typically leads to false positives. Complex
programming logic and data coming from diverse sources make the already hard
problem worse. As a result, successful execution of a real application can never
be guaranteed, and unexpected failures may happen. These failures often result
in applications throwing runtime exceptions, which if not handled correctly
may crash the application.

\ignore{ \begin{table}[t] \scriptsize \centering
\begin{tabular}{|l|r|r|}
\hline
\multicolumn{1}{|c|}{\textbf{Runtime Exception Type}} &
\multicolumn{1}{c|}{\textbf{Frequency}} & \multicolumn{1}{c|}{\textbf{\%}}\\
% \scalebox{0.83}
% {
\hline
\code{NullPointerException} & $34912$ & $54.94$ \\
\code{ClassCastException} & $7504$ & $11.81$ \\
\code{IndexOutOfBoundsException} & $6637$ & $10.44$ \\
\code{SecurityException}  & $5818$ & $9.15$ \\
\hline
\end{tabular}
\caption{Prominent runtime exceptions from
\texttt{stackoverflow}~\cite{stackoverflow}.}
\label{tab:stackoverlow}
% }
\end{table}
}


\lstset{language=Java, caption=Snippet from \code{fileUtils} class of Apache
Commons library. , label = snippet:exampleRepairing2}
\begin{figure}[t]
\centering
\begin{lstlisting}
public static String getPathNoEndSeparator
        (String filename) {
  return doGetPath(filename, 0);
}
private static String doGetPath
        (String filename, int separatorAdd) {
  if(filename == null) return null;
  int prefix = getPrefixLength(filename);
  if (prefix < 0) return null;
  int index = indexOfLastSeparator(filename);
  if ((prefix >= filename.length()) || (index < 0))
        return "";
  return filename.substring(prefix,
        index + separatorAdd);
}
\end{lstlisting}
\end{figure}

Code~\ref{snippet:exampleRepairing2} corresponds to methods from
\code{fileUtils} class of Apache Common IO library. The method
\code{getPathNoEndSeparator()} throws a \code{StringIndexOutOfBounds} exception,
which originates from statement \code{return filename.substring(prefix, index +
separatorAdd)} on line 13 when the method is called with parameter
\code{"/foo.xml"}.  Here, the value of \code{prefix} as returned by the method
\code{getPrefixLength} is 1. It fails to satisfy the constraint implied by the
program condition \code{prefix <= index + separatorAdd} for \code{substring}
method, which ensures that \code{beginIndex} cannot be greater than
\code{endIndex}. As a result, the exception is thrown.


\lstset{language=Java, caption=Patch for \code{fileUtils} class
from Apache Commons library bug., label = snippet:exampleRepairing3, firstnumber
=13}
\begin{figure}[t]
\centering
\begin{lstlisting}
String temp = null;
try {
  temp = filename.substring(prefix, index + separatorAdd);
} catch(IndexOutOfBoundsException ex) {
  int length = filename.length;
  int t = index + separatorAdd;
  temp = filename.substring(
    getStart(prefix,t,length), getEnd(prefix,t,length));
}
return temp;
\end{lstlisting}
\end{figure}

A closer inspection of this code snippet shows that the string variable
\code{filename} invokes two methods, namely \code{length} and \code{substring}
on lines $11$ and $13$ respectively. \java\ \code{String} API documentation
specifies that \code{length} does not throw any runtime exceptions. The only
exception that this invoke statement can throw is when the receiver object
referenced by \code{filename} is \code{null}. However, the check on line $7$
indicates that this situation would not arise. The method \code{substring} may
throw \code{IndexOutOfBoundsException} exception that can potentially crash the
program. A good patch to handle this failure should take into account all of
these observations. 

Code~\ref{snippet:exampleRepairing3} presents the patch automatically generated
by \tool. This patch replaces the invoke statement on line 13 in
Code~\ref{snippet:exampleRepairing2}, which is now wrapped within a
\code{try--catch} block. The \code{catch} corresponding to
\code{IndexOutOfBoundsException}
% is added on line 15
ensures that control passes
to the catch block only when the exception is thrown. Line 20 shows two method
calls namely \code{getStart} and \code{getEnd} that are inserted by \tool. These
methods, using the knowledge about the length of \code{filename} acquired with
the help of the code on line 17, compute legally correct indexes required by
\code{substring} method to satisfy the constraint related to \code{beginIndex}
and \code{endIndex}. Method \code{substring} can now regenerate the substring
ensuring that the method call will not fail. The actual patch provided by the
developers is semantically similar to the one developed by \tool, and both
versions of the program generate the exact the same output.
% for the failed test case.

Similarly, the patch developed by \tool\ for the bug depicted in
Code~\ref{snippet:exampleRepairing1} is semantically similar to the actual one
provided by the developers and is presented in
Code~\ref{snippet:exampleRepairing4}. Here the object referenced by the string
variable \code{temp} is regenerated after adjusting the offset and ensuring that
the constraint represented by the program condition \code{offset <= length}
would never be violated.


\lstset{language=java, caption=Patch for the Apache Log4j bug.,
label = snippet:exampleRepairing4, firstnumber =4}
\begin{figure}[t]
\centering
\begin{lstlisting}
try {
    temp = new String(chars, offset, length);
} catch(StringIndexOutOfBoundsException ex) {
    int i = chars.length;
    temp = new String(chars,
        IndexRepair.getStart(offset, length, i),
            IndexRepair.getEnd(offset, length, i));
}
priorVariables.add(temp);
\end{lstlisting}
\end{figure}

\note{\tool\ performs hybrid constraint analysis to produce high-quality patches.
This analysis ensures that the generated \code{String} objects obeying constraints imposed on them
by the program so as to exhibit a behavior that is close to the intended one. This ability of
\tool\ is illustrated by an example in
Code~\ref{snippet:constraintCollection}. Line numbers $3, 4, 5 $ and $7$ in
Code~\ref{snippet:constraintCollection} contain conditions associated with
\code{st}. The first three constraints can be collected and evaluated
statically and the last one dynamically as \code{userInput()} will return a
\code{String} object only at runtime. All of these constraints are stored in a
data structure called \textit{constraint store} for evaluation which is shown in
Figure~\ref{fig:constraint}. %\tool\ uses constraint store to evaluate
%which is described in Algorithm~\ref{algo:constraint}. More detailed process is
%discussed later in~\xref{sec:tool:stage2}.
The process of evaluation is described in Algorithm~\ref{algo:constraint} and
discussed in~\xref{sec:tool:stage2}. }
\note{Aritra: Can't we provide a real example here? It need not be an
 example from our study. Any real example will do.}

\lstset{language=Java, caption=Static and dynamic constraint
collection example, label = snippet:constraintCollection, firstnumber =1}
\begin{figure}[t]
\begin{lstlisting}
void foo(){
  String st = "test String";
  if(st.length == 5) {/*do something*/}
  if(st.startsWith("ab")) {/*do something*/}
  if(st.startsWith("abcd")) {/*do something*/}
  /*userInput() accepts String input from console*/
  if(st.contains(userInput())){/*do something*/}
  st = st.substring(7, 10); /*Potential failure*/
}
\end{lstlisting}
\end{figure}


\section{Problem Definition}
\label{sec:definition}

Let the behavior $\mathcal{B}$ of program $\mathcal{P}$ for input $\mathcal{I}$
be a sequence of data values $\textless b_1, \ldots, b_n \textgreater$ shared
with the environment, where these values may be used to print information on
screen, access and manipulate files and databases, and exchange data with other
processes or threads. For brevity, we assume that the program is sequential.
However, our formalization and arguments can be extended to multi-threaded
programs and their behaviors.

Consider the behavior $\mathcal{B}$ to be composed of $\mathcal{B}_c = \textless
b_{1_c}, \ldots, b_{n_c} \textgreater$ and $\mathcal{B}_n = \textless b_{1_n},
\ldots, b_{n_n} \textgreater$, where elements in $\mathcal{B}_c$ consist of
critical values that form core functionality of the program, while elements in
$\mathcal{B}_n$ are noncritical values with respect to the core functionality of
the program.
% We consider behaviors of the two programs, $\mathcal{P}$ and
% $\mathcal{P}'$, equivalent if their critical components are identical for same
% program inputs. Formally, $\mathcal{P} \equiv \mathcal{P}' \iff$
% ($\mathcal{B}_c
% = \mathcal{B}'_c \land \mathcal{I} = \mathcal{I}'$). In other words, for
% equivalent program behaviors their noncritical behaviors do not matter.
%  
% Let $\mathcal{B}_{I_f}$ be the behavior for $\mathcal{P}$ under failure input
% $\mathcal{I}_f$. and let the data element that correspond to the failure does
% not belong to $\mathcal{B}_{{I_f}_c}$. Let the element be $b_{m_n}$.
% 
% Let p be an element of P which results in a failure under cer-
% tain input I f . Let the intended behavior of P under this input be
% BI f , and let the data element that correspond to the failure does not
% belong to BI f c . Let the element be bmn .
%  
% In this work, we restrict our approach to data values that are strings.
% 
If $\mathcal{B}_{\theta}$ is the behavior for $\mathcal{P}$ under failure input
$\theta$, then our approach attempts to
% i) find a minimal $p$ that was associated with the failure,
% ii) ensures that $b_{m_n} \not\in \mathcal{B}_{{I_f}_c}$,
develop a program patch to convert $\mathcal{P}$ to $\mathcal{P}'$,
% while ensuring that:
such that 
% \begin{mybullet}
%  \item
 $\mathcal{P}'$ does not violate the core behavior of $\mathcal{P}$ under
failure, \ie\ $\mathcal{B}'_{\theta_c}$ is same as the intended behavior
$\mathcal{B}_{\theta_c}$.
% , and
%  \item
Note that resulting behavior $\mathcal{B}'_{\theta_n}$ may or may not be
equivalent to $\mathcal{B}_{\theta_n}$.
% is close to the intended
% behavior $\mathcal{B}_{\theta_n}$. In other words, the difference between
%  $\mathcal{B}'_{\theta_n}$ and 
% $\mathcal{B}_{\theta_n}$ is within limits acceptable to the programmer.
% \end{mybullet}

\note{the reviewers are not happy with the informal notion of "close to" or "acceptable".
The changes I have made still do not address this issue convincingly. If we cannot
come up with a good solution to this problem, we might say
that formal definition of acceptance criteria and its enforcement is future work. Please
comment on this. 
}

\note{I removed the offending text. Does it read better? -- md}

% \subsection{Goals}
% \label{sec:tool:goals}

In this work, we restrict our approach to string data values. We identify the
broad design goals for a technique to automatically repair malformed strings or
incorrect handling of strings as follows:

% i) identifies the statements which might be vulnerable to string-related
% errors,
% and are less critical to the functionality of the application such that
% suboptimal behavior might be acceptable,
% iii) generates patches by identifying constraints on the string data and if
% required, tweaks \code{String} API  parameters to regenerate legally correct
% string data,
% iv) optimizes the number of statements to be patched by retaining only the
% ones
% that need to be protected,

\myparagraph{(i) High patch fidelity} We require that the patched program must
preserve the intended program behavior, \ie\ the patch must be precise, and
should not induce any undesirable control flows in the repaired program. 
This goal naturally follows from the problem definition. However, we set two
more goals associated with the security and performance of the technique.

\myparagraph{(ii) Non-invasive instrumentation} We require that the technique
must ensure no side-effects (aside from optimally repairing objects) during
normal program execution, and activate patches only when the program is
guaranteed to crash.

\myparagraph{(iii) Low system overhead} We desire that the patched program must
incur no runtime overhead during normal program execution, and only negligible
overhead in case of failures.

% We next present in detail techniques and algorithms to produce patches under
% more complex scenarios.
% Our study presented in \xref{sec:results} suggests that majority of
% the string generation scenarios in practice are less complex.

\section{\tool}
\label{sec:tool}

\subsection{Goals}
\label{sec:tool:goals}

We identify the broad design goals for a technique to automatically repair
malformed strings or incorrect handling of strings as follows:

% i) identifies the statements which might be vulnerable to string-related errors,
% and are less critical to the functionality of the application such that
% suboptimal behavior might be acceptable,
% iii) generates patches by identifying constraints on the string data and if
% required, tweaks \code{String} API  parameters to regenerate legally correct
% string data,
% iv) optimizes the number of statements to be patched by retaining only the ones
% that need to be protected,  

\myparagraph{(i) High patch fidelity} We require that the patched program must
preserve the intended program behavior, \ie\ the patch must be precise and
should not induce any undesirable control flows in the repaired program.

\myparagraph{(ii) Non-invasive instrumentation} We require that the technique
must ensure no side-effects during normal program execution and activate patches
only when the program is guaranteed to crash.

\myparagraph{(iii) Low system overhead} We desire that the patched program must
incur no runtime overhead during normal program execution and only negligible
overhead in case of failures.

\subsection{Design}
\label{sec:tool:design}

\begin{figure}[t]
\centering
\includegraphics[scale=.38]{images/NewDesignDiagram.pdf}
\caption{\tool\ workflow.}
\label{fig:overallDesign}
\end{figure}

\myparagraph{\underline{Key Idea}} \tool\ leverages precise taint analysis and
call graph analysis to identify program instrumentation points, and builds upon
custom algorithms to generate targeted, high quality patches for repairing
programs with potential runtime exceptions, while still satisfying goals
mentioned in \xref{sec:tool:goals}.

Figure~\ref{fig:overallDesign} shows \tool's workflow, which involves three
main stages. First, \tool\ uses precise program analysis techniques to identify
points of interest, \ie\ string objects or API arguments that must be repaired
to prevent runtime exceptions. In the second stage, \tool\ leverages novel
custom algorithms to generate relevant patches. Specifically, \tool\ performs
intra-procedural static and dynamic analyses to identify and evaluate
constraints on the string objects under consideration. Third, \tool\ uses the
constraints evaluated in the earlier stage to programatically generate and embed
patches inside \texttt{catch} blocks to ensure that they do not get activated
during normal program execution.

\subsubsection{Precise Identification of Instrumentation Points}
\label{sec:tool:stage1}

In this stage, \tool\ leverages a combination of program analyses to accurately
determine the minimum set of points of interest where instrumentation is
required to repair. We list several techniques below that help \tool\ achieve
precision.

\myparagraph{(i) Taint analysis}: The main purpose of taint analysis is to
broadly identify which program statements can be patched (possibly even
suboptimally) without affecting the program control flow, \ie\ affect only
objects that are generated and stay within the application throughout their
lifetime. While this principle is not a binding constraint, it ensures that
\tool's repairing mechanism does not adversely affect critical program behavior.
We specify a generic set of sensitive sources and sensitive sinks for each input
program, to identify critical program paths where a repaired \code{String}
objects (and thus possibly suboptimal) must not flow. For example, \tool\ does
not repair program statements that lie along a control flow path that leads to
an I/O sink, like file system, console, network, GUI, etc.

\begin{table}[t]
\centering
\scriptsize
% \setlength{\tabcolsep}{3pt}
\begin{tabular}{l|l}
\multicolumn{1}{c|}{\textbf{Class}} & \multicolumn{1}{c}{\textbf{Source}}\\
\hline
\code{java.io.InputStream} & \code{read()}\\
\code{java.io.BufferedReader} & \code{readLine()}\\
\code{java.net.URL} & \code{openConnection()}\\
\code{java.util.Scanner} & \code{next()}\\
% \code{javax.servlet.http.HttpServletRequest} & \code{getParameter()}\\
\code{javax.servlet.ServletRequest} & \code{getParameter()}\\
\code{org.apache.http.HttpResponse} & \code{getEntity()}\\
\code{org.apache.http.util.EntityUtils} & \code{toString()}\\
\code{org.apache.http.util.EntityUtils} & \code{toByteArray()}\\
\code{org.apache.http.util.EntityUtils} & \code{getContentCharSet()}\\
\end{tabular}
\caption{Common sensitive sources in \java.}
\label{table:TaintSources}
\end{table}

\begin{table}[t]
\centering
\scriptsize
% \setlength{\tabcolsep}{3pt}
\begin{tabular}{l|l}
\multicolumn{1}{c|}{\textbf{Class}} & \multicolumn{1}{c}{\textbf{Sink}}\\
\hline
\code{java.io.FileOutputStream} & \code{write()}\\
\code{java.io.OutputStream} & \code{write()}\\
\code{java.io.PrintStream} & \code{printf()}\\
\code{java.net.Socket} & \code{connect()}\\
\code{java.io.Writer} & \code{write()}\\
\end{tabular}
\caption{Common sensitive sinks in \java.}
\label{table:TaintSinks}
\end{table}

The taint analysis module take as input the compiled byte code intended to be
repaired, and generates a control flow graph (CFG) identifying program
statements that lie along paths from sensitive sources to sensitive sinks.
Since, \tool\ targets strings in particular, it must support taint propagation
for all \java\ APIs that support string manipulation, including
\code{StringBuffer} and \code{StringBuilder}. All \code{String} objects (whether
generated or assigned) that lie along the tainted path from a sensitive source
to a sensitive sink are marked as \textit{unsafe} to patch. Subsequently, \tool\
does not repair such \code{String} objects. Figures~\ref{table:TaintSources} and
\ref{table:TaintSinks} list some common sensitive sources and sinks for several
classes in \java.

\myparagraph{(ii) Call graph analysis}:

\myparagraph{(iii) Optimizations}:


\ignore{
\paragraph{Identifying Program Statements.} We perform static taint analysis
to identify sensitive data which are leaving the system via database, network
stream, file stream or console. Providing patches to the
statements that manipulate this data would be undesirable, since
activation of the patches in case failures may allow altered sensitive data to
eventually
reach users. Hence, we only mark those program statements which do
not manipulate these data.

\paragraph{Noninvasive Patching.} In case a runtime exception that is thrown
by a statement as a result of a failure is already caught and handled in a
program,
we skip that statement from patching to avoid interfering with the results. Such
statements are identified by analyzing call-graphs and ensuring that no caller
method
in the call-chain handles the exception or its superclass. By embedding the
patches inside
\texttt{catch} blocks, we ensure that they do not get activated during normal
program execution.

\paragraph{Patch Generation.} We first perform an
intra-procedural static analysis
to identify constraints on the string objects under consideration. By
identifying
the type of exceptions that can be thrown in case of a failure, we
develop patches that
regenerate string objects by tweaking \java\ \code{String} API used in
the statements to regenerate legal string objects and by trying to solve the
constraints. In the latter case,
we evaluate the constraints statically if complete information is available at
the compilation-time. Otherwise, the analysis automatically generates
code that performs dynamic analysis to solve the constraints, and then
inserts this code in the generated patches.

\paragraph{Optimizing Instrumentation.} We perform reaching definitions
analysis to skip marked statements
if the string variables that are contained in the statements are already
patched, and the variables
are not redefined along any path that originates from the patched statement.
This analysis reduces
instrumentation points in a program.

\paragraph{Patch Precision.} The precision of a program patch is improved,
firstly, by targeting only strings
for patching which allows us to develop more specialized patches, secondly, by
patching programs very
close to the points of potential failures which avoids unnecessary patching of
other
unaffected variables and their potential
side effects, thirdly, by analyzing the types of exceptions that can be thrown
which
provides valuable insights into
origins of failures, and finally, by considering all the constraints
on the strings. This would result
in a program behavior closed to the intended one in case of a failure.

\paragraph{Reduced Overhead.} The side-effect of non-invasive patches is that
they do not interfere during
normal execution which results in no runtime overhead. Even when they get
activated in case of failures,
they still cause negligible overhead since we perform no analysis during runtime
except if required resolve the dynamic constraints.
As our study reveals~\xref{sec:evaluation} the constraints are typically few and
simple, making
the dynamic analysis light-weight.
}


\section{Implementation}
\label{sec:implementation}

We implemented a prototype of \tool\ as described in \xref{sec:design} for
repairing runtime exceptions originating from unhandled \java\ \code{String}
APIs. Our end-to-end toolchain is completely automated and was written in $x$
lines of \java. We leveraged the \soot~\cite{soot} framework for bytecode
analysis and instrumentation, and \infoflow~\cite{infoflow} for static taint
analysis.

% We have used our repairing strategy on the \java\ \code{String} API as it is one
% of the most frequent used APIs in commercially available \java\ applications and
% libraries. Another reason to choose \code{String} API was that we have found
% plenty of bugs related to \code{String} in some of the popular libraries and
% applications provided by Apache foundation, Eclipse, ASM etc.

We now briefly describe a few salient features of our implementation.

\subsection{Taint Analysis}

\infoflow\ performs its taint propagation over \code{Units}, which are \soot's
intermediate representation of the \java\ source code. We extended the
\infoflow\ framework to a) enable seamless coupling with \soot, and b) determine
whether it is safe to patch a given \soot\ \code{Unit}. Specifically, we added
a mapping that retrieves \code{Unit}s for statements to be patched given a
specified function signature. This is relevant since the same statement, say
\code{int x = 1;} has the exact same representation even if it appears in
multiple different functions. We also added a utility method to determine if a
\code{Unit} must be patched if it lies along the path between a source and sink
(recall Tables~\ref{tab:TaintSources} and ~\ref{tab:TaintSinks}) in the call
graph (as generated by \soot).

%  For the taint analysis phase we have used \soot\ \infoflow\ framework. This
%  framework requires configuration files to describe source and sink methods.
%  We
%  have identified couple of methods and some of them are tabulated in the
%  table~\ref{tab:TaintSources} and~\ref{tab:TaintSinks} respectively. We have
%  extended their InfoFlow class and added methods which store the statements in
%  a
%  \code{HashMap} object as \code{Unit}. The taint analysis phase takes two
% inputs: the jar file of the project which is to be analyzed and the
% \code{SootMethod}
%  signature of the entry point of that project.

\subsection{Constraint Analysis}
\label{subsec:constraint analysis}

Once the \infoflow\ taint analysis module identifies the \code{Units} to be
patched, \tool\ invokes its constraint analysis to determine the nature of the
patch. Specifically, patching involves a static component, which determines the
constraints by making a forward pass over the \code{Units} identified by the
taint analysis step, and a dynamic component, which is triggered if the
constraints could not be evaluated statically.

\subsubsection{Static Constraint Analysis}
\label{subsubsec:staticConstraint}

\tool\ performs static constraint analysis for all required \code{String}
objects to make the patched \code{String} as close to the ideal object. In this
phase, \tool\ analyzes all conditional statements on \code{String} literals to
statically determine a) prefix, b) suffix, and c) the length of the concerned
\code{String} literal. For example, \code{if(str.length() == 5} indicates that
for the program to enter the \code{True} branch of the conditional, \code{str}
must be of length $5$.

We collects all these information in a custom data type
and update it. For simplicity, we keep only the information such as minimum and
maximum length, set of characters which the string
may contains and set of possible prefix. With all these information, we generate 
the string object statically. The sting generation algorithm is described in
Algorithm~\ref{algo:constraint}. 

\subsubsection{Dynamic Constraint Analysis}
\label{subsubsec:dynamicConstraint}

We performed dynamic analysis in case the constraint can not be evaluated 
statically e.g. \code{if(str.contains(inputString()))}. In such cases just
before the conditional statement we instrument the bytecode with a static 
invocation which will populate the custom constraint data type and recalculate 
the string object with already existing constraints.

\subsection{String Repairing Phase}
\label{subsec:stringReepairing}

The string repairing phase is divided into two sub-phases.

\subsubsection{Detecting Potential Point of Failure}
\label{subsub:detectingFailure}

We have used specification from \java\ SE official documentation and list all
the methods which throws runtime exception. We do forward pass to see if there 
is any invocation of such methods and if we find any we then cross check it 
with the results we got from the taint analysis. We also see if there is already
some exception handling mechanism provided by the developer using the technique 
described in Section~\ref{subsec:callChainLookUp}. We detected such method calls
and wrap them in try-catch block. In the catch block we place the appropriate
exception type as provided by \java\ SE API documentation. 

\subsubsection{Catch Block Instrumentation}
\label{subsub:catchInstrumentation}

In this phase we instrument appropiate patching codes inside the catch block.
We used the static constraint evaluation of
Section~\ref{subsubsec:staticConstraint} to statically evaluate the string. In
cases there are more constraints which can't be solved statically, it would instrument
necessary method call so that the constraints would populate and get solved in runtime
~\ref{subsubsec:dynamicConstraint}. In case there is no constraint, we repair the
string in the method calls like \code{substrring}, \code{subSequence}, \code{charAt} etc.
which are dependent on the index arguments. In those cases we used
Algorithm~\ref{algo:stringPatchParametr} to repair them.


\begin{algorithm}
\scriptsize
\DontPrintSemicolon
\KwData{String object $Str$ and index set $IS$ which contains ${i}$ or
${i,j}$.}
\KwResult{Repaired index set containin ${Ri}$ or ${Ri,Rj}$ based on input $IS$}
\Begin
{
	$Length \longleftarrow$ length of $Str$\;
	\If{$Length == 0$} {
		$Ri, Rj \longleftarrow 0$\;
	} \Else {
		\If{$i \textgreater j$} {
			$Ri \longleftarrow j - 1$;
		}
		\If{$i \textgreater Lengrh$ \bf{OR} $j \textgreater Lengrh$} {
			$Ri \longleftarrow Length - 1$ or $Rj \longleftarrow Length - 1$ based on
			condition\; 
		}
		\If{$i \textless 0$ \bf{OR} $j \textless 0h$} {
			$Ri \longleftarrow 0$ or $Rj \longleftarrow 0$ based on
			condition\; 
		}		
	}	
}
\caption{String patching based on parameters passed}
\label{algo:stringPatchParametr}
\end{algorithm}
dfdf
ssfs

% 
% \subsection{Data Structures used in Various Phases}
% \label{subsec:dataStructure}
% 
% \begin{mylist}
% \item \textbf{Taint Analysis} : We kept \code{HashMap} object to store all the
% \soot\ \code{Unit} object corresponds to a \code{Boolean} value indicating
% the fact if the the \code{Unit} is safe to patch or not. In later phases we
% used this information. As we went through multiple phases of analysis and
% after each of the phases we did a \soot\ reset, we needed to make sure in the
% later phases object equality maintained for the \code{Unit} objects. To make 
% sure we kept the \code{PatchingChain} generated from the \code{Jimple} method
% \code{body} which is the representation of the control flow graph. For each of 
% the \code{Units}, me marked the position of it in the \code{patchingChain} to 
% make sure the comparison with the same \code{Unit} object in later phases also
% work properly. 
% 
% \item \textbf{Call Graph Analysis} : To detect already handled exception we
% see higher in the call chain if the call site of any one of the ancestors is
% wrapped in code{try-catch} block or not. To store and retrieve this information
% efficiently, we keep a \code{HashMap} where key is the method signature and 
% value is an \code{Object} array which contains a \code{Unit} and a \code{SootClass}.
% The \code{SootClass} object indicates the exception class by which the particular
% \code{Unit} was handled.
% 
% \item \textbf{Constraint Analysis} : We made a custom data type named \code{ConstraintDataType} 
% which contains the information of the constraint for a particular \code{String} object and also evaluate
% the sting when required. The data type has four fields of \code{Value} object which
% indicates the value corresponding to a \soot\ \code{Local} variable. These are minimum
% length, maximum length, an array of prefix and an array of contains (both \code{Character}
% or substring). To manage this data type, we used a \code{HashMap} which is indexed by the
% \code{String} object and the value is \code{ConstraintDataType}.
% 
% \end{mylist}


\subsection{Optimizations}
\label{subsec:optimizations}

\subsubsection{Call Chain Look-up for Already Handled Exception}
\label{subsubsec:callChainLookUp}

In some scenarios, the developer may put exception handling mechanism in case
there is any runtime exception. In such cases, we shouldn't do any repairing
as it may change the correct program behavior. There can be two cases.

\begin{mylist}

\item In the current method if the statement is wrapped in try-catch block. In
\soot\
the exception handling mechanism is handle by \code{Trap} class. Each
\code{Trap}
object has start, end and handler unit. From a particular \code{Unit}, we saw if
the unit belongs to any of the existing \code{Trap} and tag the \code{Unit} in
a \code{HashMap} object so that later at the repairing phase it can be exclude
from instrumentation.

\item If the exception is handled upper in the call chain, in the case we
generate
\code{CallGhaph} using the project's entry point as the entry point of that call
graph. For a method we did reverse Breadth First search (BFS) to see from which
methods it is invoked and also all of its ancestors in the call chain. From
there
we retrieve the information if any particular call sight was wrapped in
try-catch
block or not. In such case we tag the \code{Unit} in the \code{HashMap}
mentioned
before.

\end{mylist}

% \subsubsection{Minimize Patch Instrumentation}
% \label{subsubsec:minimizePatchInstrumentation}
% 
% To reduce number of instrumentation, we analyzed if all the statements which can
% throe runtime exception really requires shielding or not. After an
% instrumentation
% we observed if a particular string object is getting modified or not. In the
% case it
% is not modified and going through same string operation, it does not require any
% patching.
% 
% \subsubsection{Minimize Conflicts in Constraint Evaluation}
% \label{subsubsec:minimizeConflictsinConstraintEvaluation}
% 
% During the constraint evaluation both statically and dynamically, we
% instrumented
% methods calls which collect the information about the objects. To reduce
% conflicts at the time of evaluation
% we skipped some branches in which there are some thrown exception or error
% condition like
% \code{System.err.print()}. We only consider the conditional expressions in the
% branches having
% no exception or error pah.
% 
% \subsubsection{Minimize Constraint Instrumentation}
% \label{subsubsec:minimizeConstrintInstrumentation}
% 
% We instrument statements by which we can generate new string objects based on
% the
% constrains we observed in a forward pass of the program. In case a particular
% string object have not encountered any exception, we deferred the
% instrumentation
% until the first repairing instrumentation. There may be cases that the string
% object never went through such methods which can throw runtime exception, in
% such cases those instrumentation are not necessary.


\section{Evaluation}
\label{sec:results}


\subsection{Experimental Setup}
\label{sub:experimentalSetup}

We have looked to several bug repositories like Bugzilla, Apache issue tracker,
Eclipse project issue tracker etc and noticed number of bugs with major,
minor, critical and blocker priorities. We took choose several bugs from them
considering couple of facts
\begin{mylist}

\item \textbf{Popularity}: How much it is used among the developers and
industries,
\item \textbf{Severity} : We consider major, critical and blocker priority bugs
only considering the facts at the bug affects dependent applications and other
libraries severely.
\item \textbf{Age and state} : We consider latest bugs which were reported in
the last 5 years.
We also consider such bugs which still remains un-fixed.

We did all the experiments in a laptop pc equipped with a dual core Intel i5
2.3-2.9 GHz processor, 8 GB or RAM, Microsoft Windows 8.1 operating system, JDK
version 1.7-45 with 2 GB of allocated heap space. All the bug reproduction was done on
Eclipse Juno IDE. For static analysis and instrumentation we have used \soot\
version 2.5.0 and \soot\ \infoflow\ for static taint analysis. We have also used
java decompiler JD version 0.7.0.1. 

\end{mylist}


\subsection{Evaluation Matrices}
\label{sub:evaluationMartices}

We have conducted the evaluation based on the matrices which empirically
measures the precision and the performance of the developed tool. We measures
the precession in terms of the quality of the patch and some other criteria
listed bellow and the performance based on the time taken and the memory
footprint.

\begin{mylist}

\item \textbf{Patch similarity with the developers' patch} : This matrix is for
the qualitative analysis of our auto generated patches. By this measurement we
have not only ensure the effectiveness of the patch but also look at the
similarity of logic an patch construction with the developers' one.After the
patching we compared both the auto-generated patch code with the developer's
patching code we found in the bug repositories. In the case the bug is still
un-fixed we looked for the comments and discussion in the panel and collects
the information about the potential patch. At the first step, we visually
compared the patches to see how much they are similar and dissimilar with the
developers' patch. We than use the instrumented class files and place it to the
library archive replacing the buggy class files. We reproduced the similar test
cases of the bugs and used the automated patched version and later fixed version
of the same library to compare the results. In case the results are the primitive
or string type we have printed the output in the console and compare them. In
case the output is some complex object we compare the properties of them. Apart
from the test cases to reproduce the bug, we also ran couple of good test case
to make sure that the patch is not behaving any other way. Based on this
experiment we made a metric named \textbf{Patch Quality Index (PQI)} which
can contains three values, high, medium and low where we consider high PSI being
a good close quality patch.

\item \textbf{Auto-generated patch size and the develops' patch size} : Apart
from the qualitative analysis we have measured quantitative aspect of the
Auto-generated patches and compared their sizes with the developers' one.
Quantitative measurement comes to picture only when we are satisfied with the
qualitative measurement. We came up with a metric called \textbf{Patch Size
Index (PSI)}. In case our patch is qualitatively satisfactory and the size is
less than the developer one the we assign PSI as high. In case the size varies
in $\pm5\%$ we assign PSI as medium. If our patch is more than $5\%$ bigger than
the develops' one then PSI is assigned to low.

\item \textbf{Already handled exception} : We have analyzed the call graph to
see higher in the call chain or in the same method if a particular statement is
handled or not. In such cases we have abort our patching effort considering that
the exception is caught with exact exception type or its base type. We have also
done measurement to see if the patching actually disrupts the normal program
flow or not and we made a metric called \textbf{Program Flow Consistency Index
(PFCI)} which is calculated as
$$PFCI = \frac{Patch_{HE}}{Stmt_{HE}}$$

where $Patch_{HE}$ = Total number of patch placed in already handled exceptions,
and $Stmt_{HE}$ =  total number of statements in the program which are already
handled and $0 \le PFCI \le 1$. The lower value of $PFCI$ is desirable.


\item \textbf{Cascaded exception} : Cascaded exception can be a result of 
auto-generated patching in the case the patched objects are used as a input
to other methods and violates the specification there. This is the limitation
in our technique in the some complicated cases where it requires develops' 
attention. The limitation is due to the fact that the analysis for the repairing 
technique is a intra-procedural analysis and the constraint evaluation method is
very simple. For the constraint evaluation part, the solver is pluggable, we can
easily replace it with a third party solver. Cascaded exception can be noticed 
when the string object we are patching is generation some specific URI or driver
string which would be used to load or configure something. In such scenario, the
patch may produce some malformed string. In case there is a cascaded exception 
in string object, our analysis will take care of it, other wise it would abort. 

\item \textbf{Time} : For each of the analysis phase, we record the time taken.

\item \textbf{Memory Consumption} : Similarly we monitored memory consumption
for all of the phases in the analysis.

\end{mylist}

\begin{sidewaystable}
%\small
\begin{tabular}{@{\makebox[3em][r]{\rownumber\space}}|l|c|c|c|c|c|c|c|c|c|c|c|c|c|c}
\multicolumn{1}{c|}{\textbf{API}} &
\multicolumn{1}{c|}{\textbf{BugID}} &
\multicolumn{1}{c|}{\textbf{Priority}} &
\multicolumn{1}{c|}{\textbf{$PQI$}} &
\multicolumn{1}{c|}{\textbf{$PSI$}} &
\multicolumn{1}{c|}{\textbf{$PFCI$}} &
\multicolumn{1}{c|}{\textbf{$LOC$}} & 
\multicolumn{1}{c|}{\textbf{$IC_O$}} & %instrumentation with optimization
\multicolumn{1}{c|}{\textbf{$IC_{UO}$}} & %instrumentation without optimization
\multicolumn{1}{c|}{\textbf{$\mathcal{N}_{CG}$}} &
\multicolumn{1}{c|}{\textbf{$PF_{TA}$}} & 
\multicolumn{1}{c|}{\textbf{$PF_{CG}$}} &
\multicolumn{1}{c|}{\textbf{$PF_{CA}$}} &
\multicolumn{1}{c|}{\textbf{$PF_{IN}$}} &
\multicolumn{1}{c}{\textbf{$\mathcal{N}_{CE}$}}\\ % cascaded exception


\hline
\code{Aries} 	 	  		& \cite{ARIES1204} & Major 	& High & High & $0$ &$129$ &$42$
& $42$ & $3496$& & &&&\\
\code{Commons CLI1.x}  		& \cite{CLI46} & Major 	&  &  & & & & & & & &&&\\
\code{Commons CLI2.x}  		& \cite{CLI193} & Major 	&  &  & & & & & & &&& &\\
\code{Commons Compress}		& \cite{COMPRESS26} & Blocker &  &  & & & & & & &&& &\\
\code{Commons IO}   		& \cite{IO179}  & Major 	&  &  & & & & & & & &&&\\
\code{Commons Lang} 	  	& \cite{LANG457}& Major 	&  &  & & & & & & & &&&\\
\code{Commons Math} 	  	& \cite{MATH198} & Major 	&  &  & & & & & & & &&&\\
\code{Commons Net} 	  		& \cite{NET442} & Major   &  &  & & & & & & & &&&\\
\code{Commons VFS} 	  		& \cite{VFS338} & Major 	&  &  & & & & & & & &&&\\
\code{Eclipse AspectJ} 		& \cite{EclipseBug333066} & Major 	&  &  & & & & &
&&&&&\\
\code{Hive} 			  	&\cite{}& Major 	&  		  &  & & & & & & & &&&\\
\code{HttpClient} 	  		&\cite{HTTPCLIENT150}& Major 	&  &  & & & & & & & &&&\\
\code{jUDDI} 	  			&\cite{JUDDI292}& Major 	&  &  & & & & & & & &&&\\
\code{Log4j} 		  		&\cite{ApacheLog4jBug}& Major 	&  &  & & & & & & &&& &\\
\code{Pivot} 		  		&\cite{PIVOT533}& Major   &  &  & & & & & & &&& &\\
\code{Qpid} 			  	&\cite{}& Blocker &  &  & & & & & & & &&&\\
\code{Servicemix-soap} 		&\cite{SMXCOMP156}& Major   &  &  & & & &  & &&&& &\\
\code{SOAP} 			 	&\cite{SOAP130}& Major 	&  &  & & & & & & &&& &\\
\code{Struts2} 		  		&\cite{WW650}& Major 	&  &  & & & & & & &&& &\\
\code{Tapestry 5} 		  	&\cite{TAP51770}& Major 	&  &  & & & & &&& & & &\\
\code{Wicket} 		  		&\cite{WICKET4387}& Major 	&  &  & & & & &&& & & &\\
\code{XalanJ2} 		  		&\cite{XALANJ836}& Major 	&  &  & & &  & && &&& &\\

\end{tabular}

\caption{Experimental results}
\label{tab:results}
\end{sidewaystable}

\section{Discussion and Future Work}
\label{sec:discussion}

%% In this section we discuss the weaknesses of \tool\ and also propose remedies
%% to overcome them.

\begin{mylist}
 
 \item \myparagraph{Focus on String APIs} In its present form, \tool\ is
primarily targeted towards repairing \code{String} objects and handling API
exceptions. While this may seem to be a limitation, we believe that \tool{}'s
strength lies in the fact that it mines contextual data about runtime exceptions
related to \code{String} objects, which helps development of intelligent
patches. Further, \tool{}'s technique is generic and can be ported to other
classes of \java\ APIs. This requires extensive study of the characteristics
and constraints of other object types. We leave this extension for the future.
%\note{How can this be done. I think we need to give some intuition. -- md}

 \item \myparagraph{Patch correctness} \tool\ attempts to generate precise
patches considering the program context that avoids cascading exceptions to a
great extent and producing the intended behavior in cases of failure. However,
it sill cannot give guarantees about elimination of cascading exception,
particularly when there are heavy object dependencies in the program. In the
future, we plan to support \tool\ with program invariants that would ensure
acceptable behavior. The invariants can be specified by a programmer or can be
automatically generated with the help of training runs and later used as
assertion at the time of execution to ensure certain conditions remain true.
%\note{This is not clear at all. -- md}

 \item \myparagraph{Handling of limited constraints} \tool{}'s constraint data
store is easy to build as it captures limited number of fairly simple
\code{String} characteristics, which are subsequently used to generate patches
for \code{String} objects. This approach may not be adequate particularly if the
program contains a large number of complex constraints. The quality of \tool{}'s
patches would generally depend on the nature of its constraint solver, which is
pluggable. \tool{}'s current solver is simple and can efficiently handle a
limited number of simple constraints. A more sophisticated off-the-shelf solver
may improve the repair quality. However, the current evaluation of the tool on
several library APIs described in~\xref{sec:results} indicates that the
constraints that exist in practice are normally less complex and are limited in
number.

\end{mylist}
\section{Related Work}
\label{sec:relatedWork}

There has been considerable amount of research done in the area of automated
program repairing. The approaches that have been proposed by the researchers
broadly fall into two categories namely, static and dynamic.

The static approaches work based on the counter-example or the violated
invariants that are reported from the field. These approaches then repair the
program by automatically developing a patch and then ensure its correctness
using computationally intensive techniques such as model-checking
\cite{biere2014, wei-issta-2010}.
These techniques are effective in producing accurate patches. However, shutting
down the system to produce and apply patches is not always feasible or
desirable. To overcome these problems, several promising dynamic approaches have
been proposed. These approaches typically develop either suboptimal patches or
isolate the data structure that is damaged which allows at least part of the
system to be functional \cite{conf/issre/DemskyR03, conf/icse/DemskyR05,
conf/issta/DemskyEGMPR06}. The advantage of these approaches is that they are
light-weight and can fix the system on-the-fly. Long et al.
\cite{conf/pldi/LongSR14} have developed an approach that deals with two most
commonly observed software errors, and then suppressing the errors with the help
of a runtime that operates by first invoking a  signal handlers, and then by
running a dynamic symbolic execution to ensure no side-effects.
This approach is light-weight and like our approach fixes the errors on-the-fly
potentially allowing some sub-optimal behaviour for a finite time until the
systems self-stabilizes.
In contrast our approach targets only string objects for repairing allowing it
generate highly precise program patches which generate very few or none
cascading exceptional events and produces a program behavior which is very close
to the expected behavior under the event of crashing.
In addition, our approach is hybrid with a heavy static component which enables
all the analysis including the side-effect analysis based on a taint analysis to
perform dynamically. It incurs negligible overhead even in the event of
crashing.

%%added new
In the litarature there exists proir art where the authors used string
transformation and solving technique for repairing purpose. In the paper
\cite{Singh:2012}, the authors deals with semantic transformation of the string
like manipulating strings that need to be interpreted as more than a sequence of
characters, e.g., as a column entry from some relational table, or as some
standard data-type like date, time, currency, etc. In \cite{Gulwani:2011}, the
author designed a learning algorithm for learning a string expression that is
consistent with input output examples. The input output example is generated
from a mapping which maps a set of string to a string defining a operation like
concatenation. There are works on genetic programming technique like
\cite{LeGoues:2012Ex, LeGoues:2012, DBLP:journals/cacm/WeimerFGN10} where the
 technique generated program patch by using already existing test cases to deal
with bugs like infinite loop, null string, segmentation fault, buffer overflow
etc.

\ignore {
% % added from the mail : from PLDI author response

Automated repair of HTML generation errors in PHP applications using string
constraint solving

In this paper the authors proposed a technique to repair the auto generated
malformed HTML codes from the PHP scrips. Often the HTML codes do not have
proper tags which are silently corrected by the browser but the result is
different across browsers. The authors employed an efficient SAT solver named
Kodkod using cost optimization to find the best repair.
%%%%%%%%%%%%%%%%%%%%%%
Deep Typechecking and Refactoring

Here the authors focused on the java database API and related query string. They
proposed the solution of type errors which is caused due to the type mismatch in
the database and the type assigned in the program. The authors also proposed a
solution for refactoring where changing a class name associated with some
queries will reflect all the strings system wide.
%%%%%%%%%%%%%%%%%%%%%%%%
GenProg: A Generic Method for Automatic Software Repair

The author used genetic programming technique which is a stochastic search
method inspired by biological evolution. The technique generated program patch
by using already existing test cases to deal with bugs like infinite loop, null
string, segmentation fault, buffer overflow etc.
%%%%%%%%%%%%%%%%%%%%%%%%%
Automatic Program Repair with Evolutionary Computation

Same paper as the above (Journal version) written by same authors.
%%%%%%%%%%%%%%%%%%%%%%%%%%%%%
A Systematic Study of Automated Program Repair: Fixing 55 out of 105 Bugs for $8
Each

Extended work of the above. The paper deals with the real life feasibility if
GenProg like what fraction of the bugs it can repair and the cost associated
with it.
%%%%%%%%%%%%%%%%%%%%%%%%%%%
Automating String Processing in Spreadsheets Using Input-Output Examples

The author considered Microsoft Excel programs as the use case scenario and
identified string processing as the major class of programming problem which
includes names/phone-numbers/dates from one format to another, data cleansing,
extracting data from several text files or web pages into a single document,
etc. The author designed a learning algorithm for learning a string expression
that is consistent with input output examples. The input output example is
generated from a mapping which maps a set of string to a string defining a
operation like concatenation.
%%%%%%%%%%%%%%%%%%%%%%%%%%%%
Learning Semantic String Transformations from Examples

The authors deals with semantic transformation of the string like manipulating
strings that need to be interpreted as more than a sequence of characters, e.g.,
as a column entry from some relational table, or as some standard data-type like
date, time, currency, etc.


% %%%%%%%%%%%%%%%%%%%%%%%%%%%%%%%%%%%%%%%%%%%%%%%%%%%%%%%%%%%%%%%%%%%%%%%%%%%
Several approaches have been proposed in the past to ensure that programs can
recover from failures. Some of the approaches are based on static repairing
where the patches are synthesized automatically based on the counter examples
found in the field \cite{wei-issta-2010}.
However, it is not always desirable to shut down the system for the post-mortem
analysis and then relaunch it after fixing the defect. In order to overcome this
weakness, dynamic approaches have been proposed to deal with problems that are
related to memory, data, and incorrect programming constructs such as infinite
loops \cite{Carbin:2011, KlingMCR12, conf/sosp/PerkinsKLABCPSSSWZER09}. Some of
the approaches work either by identifying and isolating damaged data or memory
portions \cite{conf/issre/DemskyR03, conf/icse/DemskyR05,
conf/issta/DemskyEGMPR06}, or by delaying the execution until the program
self-stabilizes \cite{Eom:2012}, or by finding the alternative execution paths
\cite{PezzeRWZ11}, or by disabling suppressing signals and hoping that the
program can recover automatically from the errors \cite{conf/pldi/LongSR14}.
Static approaches strive for correctness whereas dynamic approaches are
typically optimistic and work on the assumption that some suboptimal behavior
under certain conditions is acceptable.

\myparagraph{Data Structure Repairing}
% \label{subsec:RecWorksDataStructure}
Demsky and Rinard have proposed approaches that repair data structures ~\cite{
Demsky03automaticdata, conf/issre/DemskyR03,conf/oopsla/DemskyR03,
conf/issta/DemskyEGMPR06} the authors mostly concentrated on specific
data-structures like \emph{FAT-32}, \emph{ext2}, \emph{CTAS} (a set of
air-traffic control tools developed at the NASA Ames research center) and
repairing them. The authors represented a specification language by which they
able to see consistency property these data-structure.
Given the specification, they able to detect the inconsistency of these
data-structures and repair them.
The repairing strategy involves detecting the consistency constraints for the
particular data structure, for the violation, they replace the error condition
with correct proposition. In the paper~\cite{conf/icse/DemskyR05}, the authors
Demsky et al. proposed repair strategy by goal-directed reasoning. This involves
translating the data-structure to a abstract model by a set of model definition
rules. The actual repair involves model reconstruction and statically mapped it
to a data structure update. In the paper~\cite{conf/oopsla/2007} authors
Elkarablieh et al. proposed the idea to statically analyze the data structure to
access the information like recurrent fields and local fields. They used their
technique to some well known data structures like singly linked list, sorted
list, doubly liked list, N-ary tree, AVL tree, binary search tree, disjoint set,
red-black tree, Fibonacci heap etc.

\myparagraph{Works on Software Patching}
% \label{subsec:RecWorksSoftPatch}
In their paper~\cite{conf/sosp/PerkinsKLABCPSSSWZER09}, authors Perkins et al.
presented their \emph{Clear view} system which works on windows x86 binaries
without requiring any source code. They used invariants analysis for which they
used Daikon~\cite{DBLP:journals/scp/ErnstPGMPTX07}. They mostly patched security
vulnerabilities by some candidate repair patches.

Fan Long et al. in their paper~\cite{conf/pldi/LongSR14} presented their new
system \emph{RCV} which recovers applications from divide-by-zero and
null-deference error. Their tool replaces \emph{SIGFPE} and \emph{SIGSEGV}
signal handler with its own handler. The approach simply works by assigning zero
at the time of divide-by-zero error, read zero and ignores write at the time of
null-deference error. Their implementation was on $x86$ and $x86-64$ binaries
and they also implemented a dynamic taint analysis to see the effect of their
patching until the program stabilizes which they called as \emph{error
shepherding}.

\myparagraph{Genetic Programming, Evolutionary Computation}
% \label{subsec:RecWorksGeneric}
Reserch works on program repair based on genetic programming and evolutionary
computation can be found in the paper of Forrest et al.s~\cite{conf/gecco/2009g}
and Weimer et al.~\cite{DBLP:journals/cacm/WeimerFGN10} respectively. In the
papers, the authors used genetic programming to generate and evaluate test
cases. They used their technique on the well known Microsoft Zune media player
bug causing the application to freeze up.
}




\section{Conclusion and Future Works}
\label{sec:conc}


% \acks

% \clearpage

%\raggedright
\small
%\bibliographystyle{abbrvnat}
\bibliographystyle{plain}
\bibliography{paper}

\end{document}
